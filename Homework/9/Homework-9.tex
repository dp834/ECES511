\documentclass{article}

\usepackage[margin=.75in]{geometry}
\usepackage{amsmath}
\usepackage{amssymb}
\usepackage[shortlabels]{enumitem}
\usepackage{pgfplots}
\usepackage{circuitikz}
\usepackage{float}

\author{Damien Prieur}
\title{Homework 8 \\ ECES 511}
\date{}

\begin{document}

\maketitle

\section*{Problem 1}
Given matrix $A = \begin{bmatrix} 1 & 2 \\ 2 & 1 \end{bmatrix}$, $b = \begin{bmatrix}4\\1\end{bmatrix}$.
\newline
If $Ax=b$, calculate $x = \begin{bmatrix} x_1 \\ x_2 \end{bmatrix}$ vector using psuedo inverse method manually.
\newline
$Ax=B \implies x = A^{-1}b$ if $A$ is invertible.
$$ A^{-1} = \frac{1}{det(A)} adj(A)  = \frac{1}{-3} \begin{bmatrix} 1 & -2 \\ -2 & 1 \end{bmatrix} = \frac{1}{3}\begin{bmatrix} -1 & 2 \\ 2 & -1 \end{bmatrix} $$
$$ x = \frac{1}{3}\begin{bmatrix} -1 & 2 \\ 2 & -1 \end{bmatrix} \begin{bmatrix}4\\1\end{bmatrix} = \frac{1}{3}\begin{bmatrix} -2 \\ 7 \end{bmatrix} $$
$$ x = \begin{bmatrix} -2 / 3 \\ 7 / 3 \end{bmatrix} $$

\newpage
\section*{Problem 2}
Given matrix $A = \begin{bmatrix} 1 & 2 \\ 2 & 1 \end{bmatrix}$, $b = \begin{bmatrix}4\\1\end{bmatrix}$.
\newline
If $Ax=b$, calculate $x = \begin{bmatrix} x_1 & x_2 \end{bmatrix}$ vector using SVD method manually.
\newline
Start by computing $A^TA$
$$A^TA
=
\begin{bmatrix}
5 & 4 \\
4 & 5
\end{bmatrix}
$$
Finding the singular values we get.
$$
det(A^TA - \lambda I)
=
\begin{vmatrix}
5-\lambda & 4 \\
4 & 5-\lambda
\end{vmatrix}
=
\lambda^2 -10\lambda + 25 - 16
=
\lambda^2 -10\lambda + 9
=
(\lambda - 9)(\lambda - 1)
$$
$$ \lambda_1 = 9 \quad \lambda_2 = 1 $$
$$ \sigma_1 = 3 \quad \sigma_2 = 1 $$
$$ \Sigma = \begin{bmatrix} 3 & 0 \\ 0 & 1 \end{bmatrix} $$
Getting $V^T$ we find the eigenvectors of $A^TA$
$$\lambda = \lambda_1 = 9 =
\begin{bmatrix}
-4 & 4 \\
4 & -4
\end{bmatrix}
x
= 0
\implies
q_1
=
\frac{1}{\sqrt{2}}
\begin{bmatrix}
1 \\
1
\end{bmatrix}
$$
$$\lambda = \lambda_2 = 1 =
\begin{bmatrix}
4 & 4 \\
4 & 4
\end{bmatrix}
x
= 0
\implies
q_2
=
\frac{1}{\sqrt{2}}
\begin{bmatrix}
1 \\
-1
\end{bmatrix}
$$
$$
V^T =
\frac{1}{\sqrt{2}}
\begin{bmatrix}
1 & 1 \\
1 & -1
\end{bmatrix}
$$
To find each column of $U$ we have the relationship $u_i =  \sigma_i^{-1}Av_i$
$$u_1 =
\frac{1}{3}
\begin{bmatrix}
1 & 2 \\
2 & 1
\end{bmatrix}
\begin{bmatrix}
1 \\
1
\end{bmatrix}
=
\frac{1}{3}
\begin{bmatrix}
3 \\
3
\end{bmatrix}
=
\begin{bmatrix}
1 \\
1
\end{bmatrix}
\implies
u_1 =
\frac{1}{\sqrt{2}}
\begin{bmatrix}
1 \\
1
\end{bmatrix}
$$
$$u_2 =
1
\begin{bmatrix}
1 & 2 \\
2 & 1
\end{bmatrix}
\begin{bmatrix}
1 \\
-1
\end{bmatrix}
=
\begin{bmatrix}
-1 \\
1
\end{bmatrix}
\implies
u_2 =
\frac{1}{\sqrt{2}}
\begin{bmatrix}
-1 \\
1
\end{bmatrix}
$$
$$
U =
\frac{1}{\sqrt{2}}
\begin{bmatrix}
1 & -1 \\
1 & 1
\end{bmatrix}
$$
Now that we have the SVD of A we can find the solution to $Ax = b$ with the following equation
$$ x = V \Sigma^{-1}U^Tb $$
Since $\Sigma$ is a diagonal matrix its inverse is the same as $\Sigma$ but with each term's reciprocal.
$$x =
\frac{1}{\sqrt{2}}
\begin{bmatrix}
1 & 1 \\
1 & -1
\end{bmatrix}
\begin{bmatrix}
\frac{1}{3} & 0 \\
0 & 1
\end{bmatrix}
\frac{1}{\sqrt{2}}
\begin{bmatrix}
1 & 1 \\
-1 & 1
\end{bmatrix}
\begin{bmatrix}
4 \\
1
\end{bmatrix}
$$
$$
\implies
\frac{1}{2}
\begin{bmatrix}
1 & 1 \\
1 & -1
\end{bmatrix}
\begin{bmatrix}
\frac{1}{3} & 0 \\
0 & 1
\end{bmatrix}
\begin{bmatrix}
5 \\
-3
\end{bmatrix}
=
\frac{1}{2}
\begin{bmatrix}
1 & 1 \\
1 & -1
\end{bmatrix}
\begin{bmatrix}
\frac{5}{3} \\
-3
\end{bmatrix}
=
\frac{1}{2}
\begin{bmatrix}
\frac{-4}{3} \\
\frac{14}{3}
\end{bmatrix}
=
\begin{bmatrix}
-2 / 3 \\
7 /3
\end{bmatrix}
$$



\newpage
\section*{Problem 3}
Use svd method in image compression.
\begin{enumerate}[1)]
\item Import the $M \times M$ image into MATLAB and convert into gray image. (Use \textbf{imread} and \textbf{rgb2gray} command)
\newline
\begin{figure} [H]
    \centering
    \includegraphics[width=.3\linewidth]{{images/grayscale}.png}
\end{figure}
\item Use \textbf{svd} function to extract the singular values of the image.
\newline
See matlab code.
\item Calculate the summation of all singular values.
\newline
Summation of all singular values $\approx 224450$ 
\item Take the sum of the 10 largest singular values, what is the ratio of these 10 values sum compared to the total summation?
\newline
Summation of 10 largest values $\approx 116650$
\newline
Ratio of 10 largest values $\approx 0.5197$
\item Use those 10 singular values for image reconstruction, what do you get?
\newline
\begin{figure} [H]
    \centering
    \includegraphics[width=.3\linewidth]{{images/img_reconstructed_10sv}.png}
\end{figure}
\item What about using the 50 largest singular values?
\newline
Summation of 50 largest values $\approx 169210$
\newline
Ratio of 10 largest values $\approx 0.7539$
\begin{figure} [H]
    \centering
    \includegraphics[width=.3\linewidth]{{images/img_reconstructed_50sv}.png}
\end{figure}

(\textbf{Hint}: For (5) image reconstruction, the first $N$ singular values correspond to a $M \times N$ matrix $U$, $N \times N$ matrix $S$, $N \times M$ matrix $V$, thus the reconstructed image will still be in size $M \times M$)
\end{enumerate}
This process allows for the image to be represented in a smaller size at the cost of image quality during reconstruction.
The more elements of the PCA you include the better the resulting image will be.
You can tune how good you want the image to appear based on the ration of the singular values selected over the sum of all the singular values.
A ration of $1$ will capture the entire image while a ratio of $0.50$ will capture half of the image.
For this grayscale image we originally need to send $512 x 512$ bytes of data for the grayscale image.
with the reduction to the first $n$ singular values we need to send $512 \times n + n + n \times 512 = n \times (1025)$ bytes which can lead to larger numbers if the components of the image are not well represented by the reduced svd matrix.


\end{document}
