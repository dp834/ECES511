\documentclass{article}

\usepackage[margin=.75in]{geometry}
\usepackage{amsmath}
\usepackage{amssymb}
\usepackage[shortlabels]{enumitem}
\usepackage{pgfplots}
\usepackage{circuitikz}
\usepackage{float}

\author{Damien Prieur}
\title{Homework 8 \\ ECES 511}
\date{}

\begin{document}

\maketitle

\section*{Problem 1}
Find the singular values of the matrix
$A
=
\begin{bmatrix}
1 & 1 & 0 & 1 \\
0 & 0 & 0 & 1 \\
1 & 1 & 0 & 0 \\
\end{bmatrix}
$
\newline
Singular values are the square roots of the eigenvalues of $A^TA$ in decreasing order.

$$A^T
=
\begin{bmatrix}
1 & 0 & 1 \\
1 & 0 & 1 \\
0 & 0 & 0 \\
1 & 1 & 0 \\
\end{bmatrix}
\implies
A^TA
=
\begin{bmatrix}
1 & 0 & 1 \\
1 & 0 & 1 \\
0 & 0 & 0 \\
1 & 1 & 0 \\
\end{bmatrix}
\begin{bmatrix}
1 & 1 & 0 & 1 \\
0 & 0 & 0 & 1 \\
1 & 1 & 0 & 0 \\
\end{bmatrix}
=
\begin{bmatrix}
2 & 2 & 0 & 1 \\
2 & 2 & 0 & 1 \\
0 & 0 & 0 & 0 \\
1 & 1 & 0 & 2 \\
\end{bmatrix}
$$
Now we find the eigenvalues of $A^TA$ expanding along the 3rd column.
$$
det(A^TA - \lambda I)
\implies
\begin{vmatrix}
2 -\lambda & 2 & 0 & 1 \\
2 & 2 -\lambda & 0 & 1 \\
0 & 0 &  -\lambda & 0 \\
1 & 1 & 0 & 2 -\lambda \\
\end{vmatrix}
=-\lambda
\begin{vmatrix}
2 -\lambda & 2 & 1 \\
2 & 2 -\lambda & 1 \\
1 & 1 & 2 -\lambda \\
\end{vmatrix}
=\lambda ^4-6 \lambda ^3+6 \lambda ^2
$$
$$
\lambda^2 (\lambda ^2 - 6\lambda + 6) = 0
$$
$$\lambda_1 = 0 \quad \lambda_2 = 0 \quad \lambda_3 = 3+\sqrt{3} \quad \lambda_4 = 3-\sqrt{3}$$
To get our singular values we compute $\sigma_i = \sqrt{\lambda_i}$ and order the $\sigma$'s in decreasing order.
$$\sigma_1 = \sqrt{3 + \sqrt{3}} \qquad \sigma_2 = \sqrt{3 - \sqrt{3}} \quad \sigma_3 = 0 \quad \sigma_4 = 0 $$

\section*{Problem 2}
Find the SVD of the matrix
$A
=
\begin{bmatrix}
1 & 0 & 1 & 0 \\
0 & 1 & 0 & 1
\end{bmatrix}
$
\newline
First we find the $\Sigma$ matrix by finding the singular values as we did in problem 1.
$$
A^T
=
\begin{bmatrix}
1 & 0 \\
0 & 1 \\
1 & 0 \\
0 & 1 \\
\end{bmatrix}
\implies
A^TA
=
\begin{bmatrix}
1 & 0 & 1 & 0 \\
0 & 1 & 0 & 1
\end{bmatrix}
\begin{bmatrix}
1 & 0 \\
0 & 1 \\
1 & 0 \\
0 & 1 \\
\end{bmatrix}
=
\begin{bmatrix}
1 & 0 & 1 & 0 \\
0 & 1 & 0 & 1 \\
1 & 0 & 1 & 0 \\
0 & 1 & 0 & 1
\end{bmatrix}
$$
Now we find the eigenvalues of $A^TA$ expanding along the 3rd column.
$$
det(A^TA - \lambda I)
\implies
\begin{vmatrix}
1-\lambda & 0 & 1 & 0 \\
0 & 1-\lambda & 0 & 1 \\
1 & 0 & 1-\lambda & 0 \\
0 & 1 & 0 & 1-\lambda
\end{vmatrix}
= \lambda ^4-4 \lambda ^3+4 \lambda ^2
$$
$$
\lambda^2 (\lambda-2)^2 = 0
$$
$$ \lambda_1 = 0 \quad \lambda_2 = 0 \quad \lambda_3 = 2 \quad \lambda_4 = 2 $$
To get our singular values we compute $\sigma_i = \sqrt{\lambda_i}$ and order the $\sigma$'s in decreasing order.
$$\sigma_1 = \sqrt{2} \qquad \sigma_2 = \sqrt{2} \quad \sigma_3 = 0 \quad \sigma_4 = 0 $$
To compute $\Sigma$ we place the entries on the diagonal of a matrix with the same size of $A$
$$
\Sigma
=
\begin{bmatrix}
\sqrt{2} & 0 & 0 & 0 \\
0 & \sqrt{2} & 0 & 0
\end{bmatrix}
$$
\newline
Next we look to find the $V^T$ matrix which is just any orthonormal basis of the eigenvectors of $A^TA$.
So first we find the eigenvectors of $A^TA$.
$$\lambda_1 = 0 \implies
\begin{bmatrix}
1 & 0 & 1 & 0 \\
0 & 1 & 0 & 1 \\
1 & 0 & 1 & 0 \\
0 & 1 & 0 & 1
\end{bmatrix}
x=0
\implies
q_1
=
\begin{bmatrix}
-1\\
0\\
1\\
0
\end{bmatrix}
$$
And by inspection we can also find the second eigenvector associated with $\lambda = 0$
$$
q_2
=
\begin{bmatrix}
-1\\
0\\
1\\
0
\end{bmatrix}
\qquad
q_2
=
\begin{bmatrix}
0\\
-1\\
0\\
1
\end{bmatrix}
$$
$$\lambda = 2 \implies
\begin{bmatrix}
-1 & 0 & 1 & 0 \\
0 & -1 & 0 & 1 \\
1 & 0 & -1 & 0 \\
0 & 1 & 0 & -1
\end{bmatrix}
x=0
\implies
q_3
=
\begin{bmatrix}
1\\
0\\
1\\
0
\end{bmatrix}
$$
And by inspection we can also find the second eigenvector associated with $\lambda = 2$
$$
q_3
=
\begin{bmatrix}
1\\
0\\
1\\
0
\end{bmatrix}
\qquad
q_4
=
\begin{bmatrix}
0\\
1\\
0\\
1
\end{bmatrix}
$$
We can see by inspection that all the eigenvectors are orthogonal to each other.
Now they can be normalized by dividing by their magnitude.
$$
q_1
=
\begin{bmatrix}
-\frac{\sqrt{2}}{2}\\
0\\
\frac{\sqrt{2}}{2}\\
0
\end{bmatrix}
\qquad
q_2
=
\begin{bmatrix}
0\\
-\frac{\sqrt{2}}{2}\\
0\\
\frac{\sqrt{2}}{2}
\end{bmatrix}
\qquad
q_3
=
\begin{bmatrix}
\frac{\sqrt{2}}{2}\\
0\\
\frac{\sqrt{2}}{2}\\
0
\end{bmatrix}
\qquad
q_4
=
\begin{bmatrix}
0\\
\frac{\sqrt{2}}{2}\\
0\\
\frac{\sqrt{2}}{2}
\end{bmatrix}
$$
Reordering them to match the singular values we get
$$
V
=
\begin{bmatrix}
\frac{\sqrt{2}}{2} & 0 & -\frac{\sqrt{2}}{2} & 0 \\
0 & \frac{\sqrt{2}}{2} & 0 & -\frac{\sqrt{2}}{2} \\
\frac{\sqrt{2}}{2} & 0 & \frac{\sqrt{2}}{2} & 0 \\
0 & \frac{\sqrt{2}}{2} & 0 & \frac{\sqrt{2}}{2} \\
\end{bmatrix}
$$
Lastly we find $U$ whos columns can be found by computing $\sigma_i^{-1}Av_i$ where $i$ indicates the column.
For any extra columns we need just arbitrarily extend the columns to be orthonormal using gram schmidt orthogonalization.

$$u_1 = \sigma_1^{-1}Av_1 =
\frac{1}{2}
\begin{bmatrix}
1 & 0 & 1 & 0 \\
0 & 1 & 0 & 1
\end{bmatrix}
\begin{bmatrix}
\frac{\sqrt{2}}{2}\\
0\\
\frac{\sqrt{2}}{2}\\
0
\end{bmatrix}
=
\begin{bmatrix}
\frac{\sqrt{2}}{2} \\
0
\end{bmatrix}
$$
$$u_2 = \sigma_2^{-1}Av_2 =
\frac{1}{2}
\begin{bmatrix}
1 & 0 & 1 & 0 \\
0 & 1 & 0 & 1
\end{bmatrix}
\begin{bmatrix}
0\\
\frac{\sqrt{2}}{2}\\
0\\
\frac{\sqrt{2}}{2}
\end{bmatrix}
=
\begin{bmatrix}
0 \\
\frac{\sqrt{2}}{2}
\end{bmatrix}
$$
Normalizing each column we get
$$
U
=
\begin{bmatrix}
1 & 0 \\
0 & 1
\end{bmatrix}
$$
Putting all of this together we can check the solution by checking the following relationship $A = U\Sigma V^T$
$$
U
=
\begin{bmatrix}
1 & 0 \\
0 & 1
\end{bmatrix}
\qquad
\Sigma
=
\begin{bmatrix}
\sqrt{2} & 0 & 0 & 0 \\
0 & \sqrt{2} & 0 & 0
\end{bmatrix}
\qquad
V
=
\begin{bmatrix}
\frac{\sqrt{2}}{2} & 0 & -\frac{\sqrt{2}}{2} & 0 \\
0 & \frac{\sqrt{2}}{2} & 0 & -\frac{\sqrt{2}}{2} \\
\frac{\sqrt{2}}{2} & 0 & \frac{\sqrt{2}}{2} & 0 \\
0 & \frac{\sqrt{2}}{2} & 0 & \frac{\sqrt{2}}{2} \\
\end{bmatrix}
$$
$$
A =
\begin{bmatrix}
1 & 0 \\
0 & 1
\end{bmatrix}
\begin{bmatrix}
\sqrt{2} & 0 & 0 & 0 \\
0 & \sqrt{2} & 0 & 0
\end{bmatrix}
\begin{bmatrix}
\frac{\sqrt{2}}{2} & 0 & \frac{\sqrt{2}}{2} & 0 \\
0 & \frac{\sqrt{2}}{2} & 0 & \frac{\sqrt{2}}{2} \\
-\frac{\sqrt{2}}{2} & 0 & \frac{\sqrt{2}}{2} & 0 \\
0 & -\frac{\sqrt{2}}{2} & 0 & \frac{\sqrt{2}}{2} \\
\end{bmatrix}
=
\begin{bmatrix}
1 & 0 & 1 & 0 \\
0 & 1 & 0 & 1
\end{bmatrix}
$$

\section*{Problem 3}
Given matrix
$X
=
\begin{bmatrix}
1 & 2 \\
2 & 1 \\
3 & 4 \\
4 & 3 \\
\end{bmatrix}
$
\begin{enumerate}[1)]
\item Find the eigenvalues and eigenvectors $\varphi_1$ and $\varphi_2$ of the matrix.
\newline
Singular values are the square roots of the eigenvalues of $A^TA$ in decreasing order.

$$A^T
=
\begin{bmatrix}
1 & 2 & 3 & 4 \\
2 & 1 & 4 & 3 \\
\end{bmatrix}
\implies
A^TA
=
\begin{bmatrix}
1 & 2 & 3 & 4 \\
2 & 1 & 4 & 3 \\
\end{bmatrix}
\begin{bmatrix}
1 & 2 \\
2 & 1 \\
3 & 4 \\
4 & 3 \\
\end{bmatrix}
=
\begin{bmatrix}
30 & 28 \\
28 & 30 \\
\end{bmatrix}
$$
Now we find the eigenvalues of $A^TA$ expanding along the 3rd column.
$$
det(A^TA - \lambda I)
\implies
\begin{vmatrix}
30 -\lambda & 28 \\
28 & 30 - \lambda \\
\end{vmatrix}
=(30-\lambda)(30-\lambda) - 28^2
=\lambda^2 -60\lambda + 116
$$
$$
(\lambda-58)(\lambda-2) = 0
$$
$$ \lambda_1 = 58 \quad \lambda_2 = 2 $$
Finding the eigenvectors associated with $\lambda_1$ and $\lambda_2$
$$\lambda_1 = 58 \implies
\begin{bmatrix}
-28 & 28 \\
28 & - 28 \\
\end{bmatrix}
x=0
\implies
q_1
=
\begin{bmatrix}
1\\
1
\end{bmatrix}
$$
$$\lambda_1 = 2 \implies
\begin{bmatrix}
28 & 28 \\
28 & 28 \\
\end{bmatrix}
x=0
\implies
q_2
=
\begin{bmatrix}
-1\\
1
\end{bmatrix}
$$
$$
\lambda_1 = 58 \quad \lambda_2 = 28 \quad \varphi_1 = \begin{bmatrix} 1\\ 1 \end{bmatrix} \quad \varphi_2 = \begin{bmatrix}-1 \\ 1 \end{bmatrix}
$$


\item Calculate $Z = X[\varphi_1, \varphi_2]$, what is the relationship between vectors in $Z$ and vectors in $X$?
$$
Z
=
\begin{bmatrix}
1 & 2 \\
2 & 1 \\
3 & 4 \\
4 & 3 \\
\end{bmatrix}
\begin{bmatrix}
1 & -1 \\
1 & 1 \\
\end{bmatrix}
=
\begin{bmatrix}
3 & 1 \\
3 & -1 \\
7 & 1 \\
7 & -1
\end{bmatrix}
$$
\newline
The matrix $[\varphi_1 \varphi_2]$ equals a rotation matrix of $90^\circ$.
If the matrix was normalized then the vectors in $Z$ would represent the vectors but rotated 90 degrees.
Since it isn't normalized they are also scaled by a factor of $\sqrt{2}$
\end{enumerate}


\end{document}
