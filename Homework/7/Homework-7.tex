\documentclass{article}

\usepackage[margin=.75in]{geometry}
\usepackage{amsmath}
\usepackage{amssymb}
\usepackage[shortlabels]{enumitem}
\usepackage{pgfplots}
\usepackage{circuitikz}
\usepackage{float}

\author{Damien Prieur}
\title{Homework 7 \\ ECES 511}
\date{}

\begin{document}

\maketitle

\section*{Problem 1}
Use two different methods to find the unit-step response of
$$
\mathbf{\dot{x}}
=
\begin{bmatrix}
0 & 1 \\
-2 & -2
\end{bmatrix}
\mathbf{x}
+
\begin{bmatrix}
1\\
1
\end{bmatrix}
u
$$
$$ y = \begin{bmatrix}2 & 3 \end{bmatrix} \mathbf{x} $$
\begin{enumerate}[1.]
\item Convolution
$$ \mathbf{y}(t) = \mathbf{C}\mathbf{x}(t) $$
$$ \mathbf{x}(t)
=
e^{\mathbf{At}}\mathbf{x}(0) + \int_0^t e^{\mathbf{A}(t-\tau)}\mathbf{B}\mathbf{u}(\tau) d\tau
$$
$$
e^{\mathbf{At}} =
\begin{bmatrix}
e^{-t} (\sin (t)+\cos (t)) & e^{-t} \sin (t) \\
-2 e^{-t} \sin (t) & e^{-t} (\cos (t)-\sin (t))
\end{bmatrix}
$$
Since the initial conditon is not specified I will be assuming $\begin{bmatrix} 0 \\ 0 \end{bmatrix}$.
The first part of the equation will go to zero.
$$
\int_0^t
\begin{bmatrix}
e^{\tau - t} (\sin (t-\tau)+\cos (t-\tau)) & e^{\tau - t} \sin (t-\tau) \\
-2 e^{\tau - t} \sin (t-\tau) & e^{\tau - t} (\cos (t-\tau)-\sin (t-\tau))
\end{bmatrix}
\begin{bmatrix}
1\\
1
\end{bmatrix}
d\tau
$$
$$
=
\int_0^t
\begin{bmatrix}
e^{\tau -t} (2 \sin (t-\tau )+\cos (t-\tau )) \\
e^{\tau -t} (\cos (t-\tau )-3 \sin (t-\tau ))
\end{bmatrix}
d\tau
$$
$$
=
\begin{bmatrix}
\frac{3}{2}-\frac{1}{2} e^{-t} (\sin (t)+3 \cos (t)) \\
e^{-t} (2 \sin (t)+\cos (t))-1
\end{bmatrix}
$$
$$
\implies
y
=
\begin{bmatrix} 2 & 3 \end{bmatrix}
\begin{bmatrix}
\frac{3}{2}-\frac{1}{2} e^{-t} (\sin (t)+3 \cos (t)) \\
e^{-t} (2 \sin (t)+\cos (t))-1
\end{bmatrix}
$$
$$
y(t)
=
5 e^{-t} \sin (t)
$$


\item Laplace
Find the solution in the laplace space by computing $\mathbf{\hat{y}}(s) = \mathbf{\hat{G}}(s)\mathbf{\hat{u}}(s)$.
Where $\mathbf{\hat{G}} = \mathbf{C}(s\mathbf{I}-\mathbf{A})^{-1}\mathbf{B} + D$
$$\mathbf{\hat{G}} =
\begin{bmatrix} 2 & 3 \end{bmatrix}
\left(
\begin{bmatrix}
s & 0 \\
0 & s
\end{bmatrix}
-
\begin{bmatrix}
0 & 1 \\
-2 & -2
\end{bmatrix}
\right)
^{-1}
\begin{bmatrix}
1\\
1
\end{bmatrix}
$$
$$
\left(
\begin{bmatrix}
s & 0 \\
0 & s
\end{bmatrix}
-
\begin{bmatrix}
0 & 1 \\
-2 & -2
\end{bmatrix}
\right)
^{-1}
=
\begin{bmatrix}
\frac{s+2}{s (s+2)+2} & \frac{1}{s (s+2)+2} \\
-\frac{2}{s (s+2)+2} & \frac{s}{s (s+2)+2}
\end{bmatrix}
$$
$$\mathbf{\hat{G}} = \frac{5 s}{s (s+2)+2}$$
$$\mathbf{\hat{u}} = \mathcal{L}\{u\} = \frac{1}{s} $$
$$\mathbf{\hat{y}} = \mathbf{\hat{G}}\mathbf{\hat{u}} = \frac{5 s}{s(s (s+2)+2)} = \frac{5}{(s+1)^2 + 1}$$
$$ y(t) = \mathcal{L}^{-1}\{\mathbf{\hat{y}}\} = 5 e^{t} \sin(t) $$
$$ y(t) = 5 e^{-t} \sin(t) $$
\end{enumerate}

\newpage
\section*{Problem 2}
Discretize the state equation from Problem 1 for $T = 1$ and $T = \pi$
$$
\mathbf{A}_d = e^{\mathbf{A}T}
\qquad
\mathbf{B}_d = \int_0^Te^{\mathbf{A}\alpha}d\alpha\mathbf{B}
\qquad
\mathbf{C}_d = \mathbf{C}
\qquad
\mathbf{D}_d = \mathbf{D}
$$
$$
\mathbf{A}_d =
\begin{bmatrix}
e^{-T} (\sin (T)+\cos (T)) & e^{-T} \sin (T) \\
-2e^{-T} \sin (T) & e^{-T} (\cos (T)-\sin (T))
\end{bmatrix}
$$
$$
\mathbf{B}_d =
\begin{bmatrix}
1-e^{-T} \cos (T) & \frac{1}{2}-\frac{1}{2} e^{-T} (\sin (T)+\cos (T)) \\
e^{-T} (\sin (T)+\cos (T))-1 & e^{-T} \sin (T)
\end{bmatrix}
$$
$$
\mathbf{C}_d =
\begin{bmatrix}
2 & 3
\end{bmatrix}
$$
\begin{enumerate}[a)]
\item Let $T = 1$
$$
\mathbf{A}_d =
\begin{bmatrix}
 0.508326 & 0.30956 \\
-0.61912 & -0.110794
\end{bmatrix}
$$
$$
\mathbf{B}_d =
\begin{bmatrix}
 1.04707 \\
-0.182114 \\
\end{bmatrix}
$$
$$
\mathbf{C}_d =
\begin{bmatrix}
2 & 3
\end{bmatrix}
$$


\item Let $T = \pi$
$$
\mathbf{A}_d =
\begin{bmatrix}
-0.0432139 & 0. \\
 0. & -0.0432139
\end{bmatrix}
$$
$$
\mathbf{B}_d =
\begin{bmatrix}
 1.56482 \\
-1.04321
\end{bmatrix}
$$
$$
\mathbf{C}_d =
\begin{bmatrix}
2 & 3
\end{bmatrix}
$$
\end{enumerate}
$$ x[k+1] = A_dx[k]+B_d[k] $$
$$ y[k] = C_dx[k]$$

\newpage
\section*{Problem 3}
Given the scalar system $\dot{x} = ax + br$ where $x(0)=10$
Let $ a = -2 \quad b = 5$ and $r(t)$ is the unit step, $u(t)$
\newline
Part I
\begin{itemize}[$\bullet$]
\item Using the state transition equation compute the closed solution for $x(t)$ with the given initial condition and input $u(t)$.
Use the Convolution integal form (not Laplace).
\newline
\newline
$$ x(t)
=
e^{at}x(0) + \int_0^t e^{a(t-\tau)}br(\tau) d\tau
$$
$$
x(t)
=
10e^{-2t} + \int_0^t 5e^{-2(t-\tau)} d\tau
$$
$$
x(t)
=
10e^{-2t} + \frac{5}{2}-\frac{5 e^{-2 t}}{2}
$$
$$
x(t)
=
\frac{5}{2} (3 e^{-2 t}+1)
$$
\item Validate your solution using the Laplace transform from the State transition equation and invert.
$$\mathcal{L}(\dot{x}) = \mathcal{L}(ax + br)$$
$$s\hat{x}(s) - x(0) = a\hat{x}(s) + b\frac{1}{s}$$
$$\hat{x}(s) = \frac{1}{s}\frac{b}{s-a} + \frac{x(0)}{s-a}$$
$$\mathcal{L}^{-1}(\hat{x}(s)) = x(t) = \frac{5}{2} \left(3 e^{-2 t}+1\right)$$
\newline
\newline
\end{itemize}
Part II
\indent Let $T=\frac{1}{8}$ sec
\begin{enumerate}[a)]
\item Find $A_d$
\newline
\newline
$$A_d = e^{aT} $$
$$A_d = e^{-2T} = e^{-\frac{1}{4}} \approx 0.7788 $$
\item Find $B_d$ (using both formulas given in the notes/text)
\newline
\newline
$$B_d = b\int_0^Te^{a\alpha}d\alpha = 5\int_0^\frac{1}{8} e^{-2\alpha}d\alpha = \frac{5}{2}(1-e^{\frac{1}{4}}) \approx 0.553 $$
$$B_d = A^{-1}[A_d-I]B  = \frac{b(A_d-1)}{a} = \frac{5(e^{-\frac{1}{4}}-1)}{-2} = \frac{5}{2}(1-e^{\frac{1}{4}}) \approx 0.553 $$
\item Write the discrete time state equations $x[k+1] = f(x[k],u[k])$
\newline
\newline
$$x[k+1] = A_dx[k] + B_du[k]$$
Since $u[k] = 1 \quad \forall k$
$$x[k+1] = e^{-\frac{1}{4}}x[k] + \frac{5}{2}(1-e^{\frac{1}{4}}) $$
\item Look at the Matlab function "c2d" use the "zoh" option - do you get the same results?
See also CT DT Example Week7 file in WK6 handouts
\newline
\newline
$$ A = 0.7788 \quad B = 0.553 $$
Which matches what I found.
\item Using Matlab, Excel or something else recursively compute $x[1]$ through $x[10]$ note that $u[k]=1\quad \forall k$
\newline
\newline
$$
\begin{matrix}
1 & 0 \\
2 & 0.552998 \\
3 & 0.983673 \\
4 & 1.31908 \\
5 & 1.5803 \\
6 & 1.78374 \\
7 & 1.94217 \\
8 & 2.06557 \\
9 & 2.16166 \\
10 & 2.2365
\end{matrix}
$$
\end{enumerate}



\end{document}
