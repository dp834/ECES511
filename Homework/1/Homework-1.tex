\documentclass{article}

\usepackage[margin=.75in]{geometry}
\usepackage{amsmath}
\usepackage[shortlabels]{enumitem}
\usepackage{pgfplots}

\author{Damien Prieur}
\title{Homework 1 \\ ECES 511}
\date{}

\begin{document}

\maketitle

\section*{Problem 1}
The impulse response of an ideal lowpass filter is given by
$$ g(t) = 2\omega \frac{\sin(2\omega)(t-t_0)}{2\omega(t-t_0)} $$
for all $t$, where $\omega$ and $t_0$ are constants. Is the ideal lowpass filter causal?
Is it possible to build the filter in the real world?
\newline
\newline
For the filter to be causal it must only depend on constants or things computable with $x \in (-\infty, t)$.
We expect that prior to the impulse at $t_0$ there is no response so $g(t) = 0$ for $t < t_0$ since the impulse hasn't occured.
\newline
\newline
Let $t = t_0 -1$
$$
g(t_0 - 1) = 2\omega \frac{\sin(2\omega)(t_0-1 - t_0)}{2\omega(t_0-1 - t_0)} = \sin(2\omega) \neq 0
$$
Therefore our filter is not causal and therefore not realizable.

\section*{Problem 2}
Consider a system whose input and output are related by
$$y(t) =
 \begin{cases}
    \frac{u^2(t)}{u(t-1)} & u(t-1) \neq 0 \\
    0                     & u(t-1) =  0
\end{cases}
$$
for all $t$.
Show that the system satisfies the homogeneity property, but not the additivity property.
\newline
Homogeneity:
$$
\begin{cases}
    \frac{(\alpha u(t))^2}{\alpha u(t-1)} & u(t-1) \neq 0 \\
    \alpha 0                              & u(t-1) =  0
\end{cases}
=
\begin{cases}
    \frac{\alpha u^2(t)}{u(t-1)} & u(t-1) \neq 0 \\
    0                            & u(t-1) =  0
\end{cases}
=
\alpha y(t)
$$
Additivity:
$$
\begin{cases}
    \frac{(u_1(t)+u_2(t))^2}{u_0(t-1)+u_1(t-1)} & u_1(t-1) + u_2(t-1) \neq 0 \\
    0                                           & otherwise
\end{cases}
=
\begin{cases}
    \frac{(u_1^2(t)+u_1(t)u_2(t) + u_2^2(t)}{u_0(t-1)+u_1(t-1)} & u_1(t-1) + u_2(t-1) \neq 0 \\
    0                                           & otherwise
\end{cases}
$$
$$
\neq
y_1(t) + y_2(t)
=
\begin{cases}
    \frac{u_1^2(t)}{u_1(t-1)} + \frac{u_2^2(t)}{u_2(t-1)}  & u_1(t-1) \neq 0 \quad \& \quad u_2(t-1) \neq 0\\
    0                                                      & \text{otherwise}
\end{cases}
$$

\newpage

\section*{Problem 3}
Consider a system with impuse response as shown (left).
What is the zero state response excited by the input $u(t)$ shown (right).
\newline
\begin{figure}[h!]
\centering
\begin{tikzpicture}[scale=.75, every node/.style={scale=.75},>=stealth]
    \begin{axis}[
        xmin=0 ,xmax=5,
        ymin=-2,ymax=2,
        axis x line=middle,
        axis y line=middle,
        axis line style=->,
        xlabel={$t$},
        ylabel={$g(t)$},
        ]
        \addplot[no marks,black,-] expression[domain=0:1,samples=2]{x};
        \addplot[no marks,black,-] expression[domain=1:2,samples=2]{-x+2};
        \addplot[no marks,black, dashed]coordinates {(1,0)(1,1)};
    \end{axis}
\end{tikzpicture}
\hspace{2 cm}
\begin{tikzpicture}[scale=.75, every node/.style={scale=.75},>=stealth]
    \begin{axis}[
        xmin=-1,xmax=4,
        ymin=-2,ymax=2,
        axis x line=middle,
        axis y line=middle,
        axis line style=->,
        xlabel={$t$},
        ylabel={$u(t)$},
        ]
        \addplot[no marks,black,-] expression[domain=0:1,samples=2]{1};
        \addplot[no marks,black,-] expression[domain=1:2,samples=2]{-1};
        \addplot[no marks,black,-]coordinates {(1,1)(1,-1)};
        \addplot[no marks,black,-]coordinates {(2,-1)(2,0)};
    \end{axis}
\end{tikzpicture}
\end{figure}

We have
$$
y(t) = \int_{0}^{t} g(t-\tau)u(\tau) d\tau
$$
$$
g(t)
=
\begin{cases}
    t   & 0 \leq t \leq 1 \\
    2-t & 1 \le  t \leq 2 \\
    0   & otherwise
\end{cases}
\quad
u(t)
=
\begin{cases}
    1   & 0 \leq t \leq 1 \\
    -1  & 1 \le  t \leq 2 \\
    0   & otherwise
\end{cases}
$$
Using linearity we can find $y(t)$ due to $u(t)$ for $t \in (0, 1]$ then for $t \in (1,2]$ and add the results together.
\newline
$y(t)$ Due to $u(t)$ for $t \in (0,1]$
$$
y_{(0,1]}(t)
=
\begin{cases}
    \frac{t^2}{2}                   & 0 < t \leq 1 \\
    -t^2 + 3t - \frac{3}{2}         & 1 < t \leq 2 \\
    \frac{t^2}{2}-3t+\frac{9}{2}    & 2 < t \leq 3 \\
    0                               & otherwise
\end{cases}
$$
With linearity we can use $y(t) = y_{(0,1]}(t)-y_{(0,1]}(t-1)$
$$
y(t)
=
\begin{cases}
    \frac{t^2}{2}                                                   & 0 < t \leq 1 \\
    -t^2 + 3t - \frac{3}{2} - \frac{(t-1)^2}{2}                     & 1 < t \leq 2 \\
    \frac{t^2}{2}-3t+\frac{9/2} - (-(t-1)^2+3(t-1) - \frac{3}{2})   & 2 < t \leq 3 \\
    -(\frac{(t-1)^2}{2}-3(t-1)\frac{9}{2})                          & 3 < t \leq 4 \\
    0                                                               & otherwise

\end{cases}
\quad
=
\begin{cases}
    \frac{1}{2}t^2              & 0 < t \leq 1 \\
    -\frac{3}{2}t^2 + 4t - 2    & 1 < t \leq 2 \\
    \frac{3}{2}t^2 -8t + 10     & 2 < t \leq 3 \\
    -\frac{1}{2}t^2 +4t - 8     & 3 < t \leq 4 \\
    0                           & otherwise

\end{cases}
$$
\begin{figure}[h!]
\centering
\begin{tikzpicture}[>=stealth]
    \begin{axis}[
        xmin=-.6,xmax=4.5,
        ymin=-2,ymax=2,
        axis x line=middle,
        axis y line=middle,
        axis line style=->,
        xlabel={$t$},
        ylabel={$y(t)$},
        ]
        \addplot[no marks,black,-] expression[domain=0:1,samples=20]{x^2/2};
        \addplot[no marks,black,-] expression[domain=1:2,samples=20]{-3*x^2/2+4*x-2};
        \addplot[no marks,black,-] expression[domain=2:3,samples=20]{3*x^2/2-8*x+10};
        \addplot[no marks,black,-] expression[domain=3:4,samples=20]{-x^2/2+4*x-8};
    \end{axis}
\end{tikzpicture}
\end{figure}

\end{document}

