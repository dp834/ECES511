\documentclass{article}

\usepackage[margin=.75in]{geometry}
\usepackage{amsmath}
\usepackage{amssymb}
\usepackage[shortlabels]{enumitem}
\usepackage{pgfplots}
\usepackage{circuitikz}
\usepackage{float}

\author{Damien Prieur}
\title{Homework 5 \\ ECES 511}
\date{}

\begin{document}

\maketitle

\noindent
For the Following problems, \textbf{manually} calculate the expression for the response then \textbf{validate} your answer with MATLAB Simulation.
Ex. If you get the response of the system $y(t) = e^{-t}$, please build your system in MATLAB and compare the output obtained from the system with $y(t) = e^{-t}$ in one plot for verification.
\section*{Problem 1}
Find the response of the output variable
$$ y = 2x_1+x_2 $$
in the system described by state equations
$$ \dot{x}_1 = -2x_1 + u $$
$$ \dot{x}_2 = x_1 - x_2 $$
for a constant input $u(t)=5$ for $t>0$, i $x_1(0)=0$, and $x_2=0$.
\newline
\newline
First we rewrite the equation to be in the form
$$ \dot{\mathbf{x}}(t)= \mathbf{A}\mathbf{x}(t) + \mathbf{B}\mathbf{u}(t) $$
$$ \dot{\mathbf{y}}(t)= \mathbf{C}\mathbf{x}(t) + \mathbf{D}\mathbf{u}(t) $$
$$
\mathbf{A}
=
\begin{bmatrix}
-2 &  0 \\
 1 & -1
\end{bmatrix}
\qquad
\mathbf{B}
=
\begin{bmatrix}
 1  \\
 0
\end{bmatrix}
\qquad
\mathbf{C}
=
\begin{bmatrix}
 2 & 1 \\
\end{bmatrix}
\qquad
\mathbf{D}
=
\begin{bmatrix}
 0 & 0 \\
\end{bmatrix}
$$
$$
\mathbf{x}(t)
=
e^{\mathbf{At}}\mathbf{x}(0) + \int_0^t e^{\mathbf{A}(t-\tau)}\mathbf{B}\mathbf{u}(\tau) d\tau
$$
$$
e^{\mathbf{At}}
=
\begin{bmatrix}
e^{-2t} & 0 \\
e^{-t}-e^{-2t} & e^{-t}
\end{bmatrix}
$$
$$
e^{\mathbf{A}(t-\tau)}\mathbf{B}\mathbf{u}(\tau)
=
\begin{bmatrix}
5e^{-2(t-\tau)} \\
5e^{-(t-\tau)}-5e^{-2(t-\tau)}
\end{bmatrix}
$$
Since $x(0)=0$ we only need to evaluate the integral.
$$
x(t)
=
\int_0^t
\begin{bmatrix}
5e^{-2(t-\tau)} \\
5e^{-(t-\tau)}-5e^{-2(t-\tau)}
\end{bmatrix}
d\tau
=
\frac{5}{2}
\begin{bmatrix}
1 - e^{-2t} \\
e^{-2t}(e^{t}-1)^2
\end{bmatrix}
$$
Since $\mathbf{D}$ is zero we can find $\mathbf{y}$ with $\mathbf{y} = \mathbf{C}x(t)$
$$\mathbf{y}(t) = \mathbf{C}x(t) = \frac{5}{2}(3 - e^{-2t} - 2e^{-t})$$
$$\mathbf{y}(t) = \frac{5}{2}(3 - e^{-2t} - 2e^{-t})$$
\begin{figure} [H]
    \centering
    \includegraphics[width=.8\linewidth]{{images/p1_predicted_vs_simulation}.png}
\end{figure}


\newpage
\section*{Problem 2}
\begin{figure}[H]
\centering
\begin{circuitikz}
\draw
    (0,0) to[american voltage source, l=$V_s(t)$] (0,4)
    to[resistor,  l=$R$] (2,4)
    to[capacitor, l=$C$] (4,4)
    to[inductor,  l=$L$] (4,0)
    -- (0,0)
    ;
\end{circuitikz}
\end{figure}
Given the standard RLC circuit (R = $1.2$, C = $1$, L = $0.2$).
To build the state space equations, we choose capacitor voltage $v_c(t)=x_1(t)$ and inductor current $i_l(t)=x_2(t)$ as state variables and $v_c(t)$ as output.
\newline
\begin{enumerate}[1)]
\item Find the state space equation for the RLC system
\newline
We can start by expressing our state variables using Kirchoff's Voltage law.
$$ V_s(t) = V_r(t) + V_c(t) + V_l(t) $$
$$ V_s(t) = iR + V_c(t) + L\frac{di}{dt} $$
$$ \frac{dV_c}{dt} = \frac{i}{C}  $$
Rewriting these in terms of our state variables we get the state equation.
$$ \dot{x}_1 = \frac{x_2}{C} $$
$$ \dot{x}_2 = \frac{1}{L}(V_s(t) - x_2R - x_1) $$
$$ y = x_1$$
$$
\mathbf{A}
=
\begin{bmatrix}
 0 &  \frac{1}{C} \\
 -\frac{1}{L} & -\frac{R}{L}
\end{bmatrix}
\qquad
\mathbf{B}
=
\begin{bmatrix}
 0  \\
\frac{1}{L}
\end{bmatrix}
\qquad
\mathbf{C}
=
\begin{bmatrix}
 1 & 0 \\
\end{bmatrix}
\qquad
\mathbf{D}
=
\begin{bmatrix}
 0 & 0 \\
\end{bmatrix}
$$
Plugging in our values
$$
\mathbf{A}
=
\begin{bmatrix}
 0 &  1 \\
 -5 & -6
\end{bmatrix}
\qquad
\mathbf{B}
=
\begin{bmatrix}
 0  \\
 5
\end{bmatrix}
\qquad
\mathbf{C}
=
\begin{bmatrix}
 1 & 0 \\
\end{bmatrix}
\qquad
\mathbf{D}
=
\begin{bmatrix}
 0 & 0 \\
\end{bmatrix}
$$


\item If $v_c(0)=0$,$i_l(0)=5$,$v_s(t)=0$, find the $\mathbf{x}(t)$ for the system
\newline
$$
\mathbf{x}(t)
=
e^{\mathbf{At}}\mathbf{x}(0) + \int_0^t e^{\mathbf{A}(t-\tau)}\mathbf{B}\mathbf{u}(\tau) d\tau
$$
$$ x(0) = \begin{bmatrix} 0 \\ 5 \end{bmatrix} $$
$$
e^{\mathbf{A}t}
=
\begin{bmatrix}
.25 e^{-1t}-0.25 e^{-5t} & 0.25 e^{-1t}-0.25 e^{-5t} \\
 1.25 e^{-5t}-1.25 e^{-1t} & 1.25 e^{-5t}-0.25 e^{-1t} \\
\end{bmatrix}
$$
$$
e^{\mathbf{A}t}x(0)
=
\begin{bmatrix}
1.25 e^{-1t}-1.25 e^{-5t} \\
6.25 e^{-5t}-1.25 e^{-1t} \\
\end{bmatrix}
$$
$$
\mathbf{u}=0 \implies
e^{\mathbf{A}(t-\tau)}\mathbf{B}\mathbf{u}
=
0
$$
$$
\implies
\mathbf{x}(t)
=
\begin{bmatrix}
1.25 e^{-1t}-1.25 e^{-5t} \\
6.25 e^{-5t}-1.25 e^{-1t} \\
\end{bmatrix}
$$
\begin{figure} [H]
    \centering
    \includegraphics[width=.8\linewidth]{{images/p2_2_predicted_vs_simulation}.png}
\end{figure}
\item If $v_c(0)=0$,$i_l(0)=0$,$v_s(t)=5$, find the $y(t)$ for the system
\newline
We have mostly the same situation just a few parameters have changed.
$$ x(0) = \begin{bmatrix} 0 \\ 0 \end{bmatrix} \qquad v_s \neq 0 $$
So we must evaluate the convolution.
$$
\int_0^t e^{\mathbf{A}(t-\tau)}\mathbf{B}\mathbf{u}(\tau) d\tau
=
\begin{bmatrix}
5 +                1.25 e^{-5t} - 6.25 e^{-1t} \\
-1.7763*10^{-15}  -6.25 e^{-5t} + 6.25 e^{-1t}
\end{bmatrix}
=x(t)
$$
And since the initial condition is zero the state variables are just the convolution.
$$y(t) = \mathbf{C}x(t)$$
$$y(t) = 5 + 1.25 e^{-5t} - 6.25 e^{-1t} $$
\begin{figure} [H]
    \centering
    \includegraphics[width=.8\linewidth]{{images/p2_3_predicted_vs_simulation}.png}
\end{figure}
\end{enumerate}

\newpage
\section*{Problem 3}
Consider the system
$$ \ddot{x} + 4\dot{x} +4x=u $$
\begin{enumerate}[1)]
\item Find the state space model for the system
\newline
Let $x_1 = x$ and $x_2 = \dot{x}$
$$
\dot{x}
=
\begin{bmatrix}
\dot{x}_1\\
\dot{x}_2
\end{bmatrix}
=
\begin{bmatrix}
x_2 \\
-4x_2 - 4 x_1 + u
\end{bmatrix}
$$
$$
\mathbf{A}
=
\begin{bmatrix}
0 & 1 \\
-4 & -4
\end{bmatrix}
\qquad
\mathbf{B}
=
\begin{bmatrix}
0 \\
1
\end{bmatrix}
\qquad
\mathbf{C}
=
\begin{bmatrix}
1 & 0
\end{bmatrix}
\qquad
\mathbf{D}
=
\begin{bmatrix}
0 & 0
\end{bmatrix}
$$
\item Find the transfer function for the system. (Hint: you can use $\mathbf{C}(s\mathbf{I}-\mathbf{A}^{-1})\mathbf{B}+D$)
\newline
$$
s\mathbf{I}-\mathbf{A}
=
\begin{bmatrix}
s  & -1\\
4 & s+4
\end{bmatrix}
\implies
(s\mathbf{I}-\mathbf{A})^{-1}
=
\frac{1}{s^2 +4s +4}
\begin{bmatrix}
s+4 & 1\\
-4   & s
\end{bmatrix}
$$
$$
\mathbf{\hat{x}}(s)
=
\frac{1}{s^2 +4s +4}
\begin{bmatrix}
1 & 0
\end{bmatrix}
\begin{bmatrix}
s+4 & 1\\
-4   & s
\end{bmatrix}
\begin{bmatrix}
0\\
1
\end{bmatrix}
=
\frac{1}{s^2 +4s +4}
$$
$$
\mathbf{\hat{x}}(s)
=
\frac{1}{s^2 +4s +4}
$$
\item Suppose $u(t) = e^{-2t}\sin{(t)}$ for $t \geq 0$, and $\ddot{x}(0) = \dot{x}(0) = 0$. Find the solution $x(t)$
\newline
We want to use $\mathbf{\hat{y}}(s) = \mathbf{\hat{G}}(s)\mathbf{\hat{u}}(s) $ where $\mathbf{\hat{G}}(s)$ is what we computed in part 2.
First we must find the input's representation in the laplace space.
$$
\mathbf{\hat{u}}(s)
=
\mathcal{L}\{e^{-2t}\sin{(t)}\}
=
\frac{1}{s^2+4s+5}
\quad t> 0
$$
$$
\mathbf{\hat{y}}(s)
=
\frac{1}{s^2+4s+5}
\frac{1}{s^2 +4s +4}
=
\frac{1}{(2+s)^2(5+4s+s^2)}
$$
$$
y(t)
=
\mathcal{L}^{-1}(\mathbf{\hat{y}})
=
e^{-2t}(t-\sin{(t)})
$$
\begin{figure} [H]
    \centering
    \includegraphics[width=.8\linewidth]{{images/p3_predicted_vs_simulation}.png}
\end{figure}


\end{enumerate}
\end{document}
