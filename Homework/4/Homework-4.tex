\documentclass{article}

\usepackage[margin=.75in]{geometry}
\usepackage{amsmath}
\usepackage{amssymb}
\usepackage[shortlabels]{enumitem}
\usepackage{pgfplots}

\author{Damien Prieur}
\title{Homework 4 \\ ECES 511}
\date{}

\begin{document}

\maketitle

\section*{Problem 1}
Given
$$
\mathbf{A}
=
\begin{bmatrix}
1 & 1 & 0 \\
0 & 0 & 1 \\
0 & 0 & 1 \\
\end{bmatrix}
$$
find $\mathbf{A}^{10}$, $\mathbf{A}^{103}$,and $e^{\mathbf{A}t}$
\begin{itemize}
\item $\mathbf{A}^{10}$
\newline
Let $f(\lambda) = \lambda^{10}$
$$\Delta(A) = (\lambda-1)(\lambda(\lambda-1)) = \lambda(\lambda-1)^2 $$
$$ \lambda_1 = 0 \qquad \lambda_2 = 1 \qquad \lambda_3 = 1 $$
Let $h(\lambda) = \beta_0 + \beta_1\lambda + \beta_2\lambda^2$
\newline
With $\lambda_1$ and knowing it's multiplicity is $1$ we have one equation
$$ h(\lambda_1) = f(\lambda_1) \implies \beta_0 + \beta_1(0) + \beta_2(0) = 0^{10} \implies \beta_0 = 0 $$
With $\lambda_2$ and knowing it's multiplicity is $2$ we have two equations
$$ h(\lambda_2) = f(\lambda_2) \implies \beta_0 + \beta_1(1) + \beta_2(1)^2 = 1^{10} \implies \beta_1 + \beta_2 = 1 $$
$$ h'(\lambda_2) = f'(\lambda_2) \implies \beta_1 + 2\beta_2(1) = (10)1^{9} \implies \beta_1 + 2\beta_2 = (10) $$
$$ \beta_0 = 0 \qquad \beta_1 = -8 \qquad \beta_2 = 9 $$
$$ \mathbf{A}^{10} = f(\mathbf{A}) = h(\mathbf{A}) = \beta_2\mathbf{A}^2 + \beta_1\mathbf{A} = 9\mathbf{A}^2 - 8\mathbf{A} $$
$$
\mathbf{A}^2
=
\begin{bmatrix}
1 & 1 & 1 \\
0 & 0 & 1 \\
0 & 0 & 1 \\
\end{bmatrix}
$$
$$
\mathbf{A}^{10}
=
\begin{bmatrix}
9 & 9 & 9 \\
0 & 0 & 9 \\
0 & 0 & 9 \\
\end{bmatrix}
-
\begin{bmatrix}
8 & 8 & 0 \\
0 & 0 & 8 \\
0 & 0 & 8 \\
\end{bmatrix}
=
\begin{bmatrix}
1 & 1 & 9 \\
0 & 0 & 1 \\
0 & 0 & 1 \\
\end{bmatrix}
$$
$$
\mathbf{A}^{10}
=
\begin{bmatrix}
1 & 1 & 9 \\
0 & 0 & 1 \\
0 & 0 & 1 \\
\end{bmatrix}
$$
\item $\mathbf{A}^{103}$
\newline
Let $f(\lambda) = \lambda^{103}$
$$\Delta(A) = (\lambda-1)(\lambda(\lambda-1)) = \lambda(\lambda-1)^2 $$
$$ \lambda_1 = 0 \qquad \lambda_2 = 1 \qquad \lambda_3 = 1 $$
Let $h(\lambda) = \beta_0 + \beta_1\lambda + \beta_2\lambda^2$
\newline
With $\lambda_1$ and knowing it's multiplicity is $1$ we have one equation
$$ h(\lambda_1) = f(\lambda_1) \implies \beta_0 + \beta_1(0) + \beta_2(0) = 0^{103} \implies \beta_0 = 0 $$
With $\lambda_2$ and knowing it's multiplicity is $2$ we have two equations
$$ h(\lambda_2) = f(\lambda_2) \implies \beta_0 + \beta_1(1) + \beta_2(1)^2 = 1^{103} \implies \beta_1 + \beta_2 = 1 $$
$$ h'(\lambda_2) = f'(\lambda_2) \implies \beta_1 + 2\beta_2(1) = (103)1^{102} \implies \beta_1 + 2\beta_2 = (103) $$
$$ \beta_0 = 0 \qquad \beta_1 = -101 \qquad \beta_2 = 102  $$
$$ \mathbf{A}^{10} = f(\mathbf{A}) = h(\mathbf{A}) = \beta_2\mathbf{A}^2 + \beta_1\mathbf{A} = 102\mathbf{A}^2 - 101\mathbf{A} $$
$$
\mathbf{A}^2
=
\begin{bmatrix}
1 & 1 & 1 \\
0 & 0 & 1 \\
0 & 0 & 1 \\
\end{bmatrix}
$$
$$
\mathbf{A}^{103}
=
\begin{bmatrix}
102 & 102 & 102 \\
0 & 0 & 102 \\
0 & 0 & 102 \\
\end{bmatrix}
-
\begin{bmatrix}
101 & 101 & 0 \\
0 & 0 & 101 \\
0 & 0 & 101 \\
\end{bmatrix}
=
\begin{bmatrix}
1 & 1 & 102 \\
0 & 0 & 1 \\
0 & 0 & 1 \\
\end{bmatrix}
$$
$$
\mathbf{A}^{103}
=
\begin{bmatrix}
1 & 1 & 102 \\
0 & 0 & 1 \\
0 & 0 & 1 \\
\end{bmatrix}
$$
\item $e^{\mathbf{A}t}$
\newline
Let $f(\lambda) = e^{\lambda t}$
$$\Delta(A) = (\lambda-1)(\lambda(\lambda-1)) = \lambda(\lambda-1)^2 $$
$$ \lambda_1 = 0 \qquad \lambda_2 = 1 \qquad \lambda_3 = 1 $$
Let $h(\lambda) = \beta_0 + \beta_1\lambda + \beta_2\lambda^2$
\newline
With $\lambda_1$ and knowing it's multiplicity is $1$ we have one equation
$$ h(\lambda_1) = f(\lambda_1) \implies \beta_0 + \beta_1(0) + \beta_2(0) = e^{0t} \implies \beta_0 = 1 $$
With $\lambda_2$ and knowing it's multiplicity is $2$ we have two equations
$$ h(\lambda_2) = f(\lambda_2) \implies \beta_0 + \beta_1(1) + \beta_2(1)^2 = e^{t} \implies \beta_0 + \beta_1 + \beta_2 = e^{t} $$
$$ h'(\lambda_2) = f'(\lambda_2) \implies \beta_1 + 2\beta_2(1) = te^{t} \implies \beta_0 + \beta_1 + 2\beta_2 = te^{t} $$
$$ \beta_0 = 1 \qquad \beta_1 = 2e^t-te^t-1 \qquad \beta_2 = te^t - e^t $$
$$ e^{\mathbf{A}t} = f(\mathbf{A}) = h(\mathbf{A}) = \beta_2\mathbf{A}^2 + \beta_1\mathbf{A} + \beta_0 = (te^t - e^t)\mathbf{A}^2 + (2e^t-te^t-1)\mathbf{A} + \mathbf{I} $$
$$
\mathbf{A}^2
=
\begin{bmatrix}
1 & 1 & 1 \\
0 & 0 & 1 \\
0 & 0 & 1 \\
\end{bmatrix}
$$
$$
e^{\mathbf{A}t}
=
(te^t - e^t)
\begin{bmatrix}
1 & 1 & 1 \\
0 & 0 & 1 \\
0 & 0 & 1 \\
\end{bmatrix}
+
(2e^t-te^t-1)
\begin{bmatrix}
1 & 1 & 0 \\
0 & 0 & 1 \\
0 & 0 & 1 \\
\end{bmatrix}
+
\begin{bmatrix}
1 & 0 & 0 \\
0 & 1 & 0 \\
0 & 0 & 1 \\
\end{bmatrix}
=
\begin{bmatrix}
e^t & e^t-1 & te^t-e^t+1\\
0 & 1 & e^t-1 \\
0 & 0 & e^t  \\
\end{bmatrix}
$$
$$
e^{\mathbf{A}t}
=
\begin{bmatrix}
e^t & e^t-1 & te^t-e^t+1\\
0 & 1 & e^t-1 \\
0 & 0 & e^t  \\
\end{bmatrix}
$$
\end{itemize}

\section*{Problem 2}
Find the least square solutions of $\mathbf{A}x=\mathbf{b}$, where
$$
\mathbf{A}
=
\begin{bmatrix}
2 & 0  \\
-1 & 1 \\
0 & 2  \\
\end{bmatrix}
\qquad
\mathbf{b}
=
\begin{bmatrix}
1  \\
0  \\
-1 \\
\end{bmatrix}
$$
What is the quantity being minimized?
\newline
\newline
We can manipulate the original equation to get
$$ x = (\mathbf{A}^T\mathbf{A})^{-1}\mathbf{A}^T\mathbf{b} $$
This equation looks to minimize $(\hat{\mathbf{b}} - \mathbf{b})^2$.
Where $\hat{\mathbf{b}} = A(\mathbf{A}^T\mathbf{A})^{-1}\mathbf{A}^T\mathbf{b}$ or what our mapping projects $\mathbf{A}$ to.
$$
\mathbf{A}^T\mathbf{A}
=
\begin{bmatrix}
2 & -1 & 0 \\
0 &  1 & 2
\end{bmatrix}
\begin{bmatrix}
2 & 0  \\
-1 & 1 \\
0 & 2  \\
\end{bmatrix}
=
\begin{bmatrix}
5 & -1 \\
-1 & 5 \\
\end{bmatrix}
$$
$$
(\mathbf{A}^T\mathbf{A})^{-1}
=
\frac{1}{|\mathbf{A}^T\mathbf{A}|}adj(\mathbf{A}^T\mathbf{A})
=\frac{1}{25-1}
\begin{bmatrix}
5 & 1 \\
1 & 5 \\
\end{bmatrix}
$$
$$
x =
\frac{1}{24}
\begin{bmatrix}
5 & 1 \\
1 & 5 \\
\end{bmatrix}
\begin{bmatrix}
2 & -1 & 0 \\
0 &  1 & 2
\end{bmatrix}
\begin{bmatrix}
1  \\
0  \\
-1 \\
\end{bmatrix}
=
\frac{1}{24}
\begin{bmatrix}
5 & 1 \\
1 & 5 \\
\end{bmatrix}
\begin{bmatrix}
2 \\
-2
\end{bmatrix}
=
\frac{1}{24}
\begin{bmatrix}
8 \\
-8
\end{bmatrix}
=
\begin{bmatrix}
\frac{1}{3} \\
-\frac{1}{3}
\end{bmatrix}
$$
How close are we
$$\mathbf{A}x=
\begin{bmatrix}
2 & 0  \\
-1 & 1 \\
0 & 2  \\
\end{bmatrix}
\begin{bmatrix}
\frac{1}{3} \\
-\frac{1}{3}
\end{bmatrix}
=
\begin{bmatrix}
\frac{2}{3} \\
-\frac{2}{3} \\
-\frac{2}{3}
\end{bmatrix}
$$
So our total error is $\frac{1}{3}^2 + (-\frac{2}{3})^2 + (-\frac{1}{3})^2 = \frac{2}{3}$


\section*{Problem 3}
Suppose that we have measured three data points
$$ (0,6), \quad (1,0), \quad (2,0)$$
and our model is linear, compute the line of best fit by the method of least squares.
\newline
\newline
Create a system of equations to find a solution for our line which has a form $y = ax+b$
$$
\mathbf{A}
\begin{bmatrix}
a\\
b\\
\end{bmatrix}
=
\mathbf{b}
$$
Each row in $\mathbf{A}$ and $\mathbf{b}$ will correspond to one data point from the three inputs.
The first column of $\mathbf{A}$ should correspond to the $x$ inputs that get multiplied by a, so the first number in each pair.
The second column of $\mathbf{A}$ will contain all $1$'s since all terms must add the $b$ term to be a line.
The $\mathbf{b}$ will contain the expected output that we hope to match for each data point.
$$
\mathbf{A}
=
\begin{bmatrix}
0 & 1 \\
1 & 1 \\
2 & 1 \\
\end{bmatrix}
\qquad
\mathbf{b}
\begin{bmatrix}
6 \\
0 \\
0 \\
\end{bmatrix}
$$
Using the same process as in Problem 2 we can find the values for $a$ and $b$ by solving this equation
$$
\begin{bmatrix}
a \\
b \\
\end{bmatrix}
=(\mathbf{A}^T\mathbf{A})^{-1}\mathbf{A}^T\mathbf{b}
$$
$$
\mathbf{A}^T\mathbf{A}
=
\begin{bmatrix}
5 & 3 \\
3 & 3
\end{bmatrix}
\qquad
(\mathbf{A}^T\mathbf{A})^{-1}
=
\frac{1}{6}
\begin{bmatrix}
3 & -3 \\
-3 & 5
\end{bmatrix}
$$
$$
\begin{bmatrix}
a \\
b \\
\end{bmatrix}
=
\begin{bmatrix}
-3 \\
5
\end{bmatrix}
$$
The equation of the best fit line is
$$ y = -3x + 5 $$

\section*{Problem 4}
Find the minimum polynomial for the matrix $\begin{bmatrix} 1 & 0 & 0 \\ 0 & 0 & -2 \\ 0 & 1 & 3 \\ \end{bmatrix}$
\newline
\newline
We can first start with the characteristic polynomial
$$
(\lambda-1)*(\lambda(\lambda-3) + 2)
=
(\lambda - 1)(\lambda^2-3\lambda + 2);
=
(\lambda-1)(\lambda - 2)(\lambda - 1);
$$
We have $3$ roots and we want the least common multiple of the roots we get that the polynomial is
$$\mu(X)=(X-1)(X-2) = X^2 -3X +2$$

\section*{Problem 5}
Find the parabola that best approximates the data points,
$$ (-1,\frac{1}{2}), \quad (1, -1), \quad (2, -\frac{1}{2}), \quad (3,2) $$
What is the quantity being minimized?
\newline
\newline
Create a system of equations to find a solution for our line which has a form $y = ax^2+bx+c$
$$
\mathbf{A}
\begin{bmatrix}
a\\
b\\
c\\
\end{bmatrix}
=
\mathbf{b}
$$
Each row in $\mathbf{A}$ and $\mathbf{b}$ will correspond to one data point from the three inputs.
\newline
The first column of $\mathbf{A}$ should correspond to the $x^2$ inputs that get multiplied by a, so the square of the first number in each pair.
\newline
The second column of $\mathbf{A}$ should correspond to the $x$ inputs that get multiplied by b, so the first number in each pair.
\newline
The second column of $\mathbf{A}$ will contain all $1$'s since all terms must add the $c$ term to be a line.
The $\mathbf{b}$ will contain the expected output that we hope to match for each data point.
$$
\mathbf{A}
=
\begin{bmatrix}
(-1)^2 & -1 & 1 \\
1^2 & 1 & 1 \\
2^2 & 2 & 1 \\
3^2 & 3 & 1 \\
\end{bmatrix}
=
\begin{bmatrix}
1 & -1 & 1 \\
1 & 1 & 1 \\
4 & 2 & 1 \\
9 & 3 & 1 \\
\end{bmatrix}
\qquad
\mathbf{b}
\begin{bmatrix}
\frac{1}{2} \\
-1 \\
-\frac{1}{2} \\
2 \\
\end{bmatrix}
$$
Using the same process as in Problem 2 we can find the values for $a$, $b$, and $c$ by solving this equation
$$
\begin{bmatrix}
a \\
b \\
c \\
\end{bmatrix}
=(\mathbf{A}^T\mathbf{A})^{-1}\mathbf{A}^T\mathbf{b}
$$
$$
\mathbf{A}^T\mathbf{A}
=
\begin{bmatrix}
99 & 35 & 15 \\
35 & 15 &  5 \\
15 &  5 &  4 \\
\end{bmatrix}
\qquad
(\mathbf{A}^T\mathbf{A})^{-1}
=
\frac{1}{440}
\begin{bmatrix}
 35 & -65 & -50 \\
-65 & 171 &  30 \\
-50 &  30 & 260 \\
\end{bmatrix}
$$
$$
\begin{bmatrix}
a \\
b \\
c \\
\end{bmatrix}
=
\frac{1}{440}
\begin{bmatrix}
 265 \\
-379 \\
-410
\end{bmatrix}
$$
The equation of the best fit line is
$$ y = \frac{53}{88}x^2 + \frac{-379}{440}x + \frac{-41}{44} $$
\end{document}
