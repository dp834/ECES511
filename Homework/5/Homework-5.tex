\documentclass{article}

\usepackage[margin=.75in]{geometry}
\usepackage{amsmath}
\usepackage{amssymb}
\usepackage[shortlabels]{enumitem}
\usepackage{pgfplots}

\author{Damien Prieur}
\title{Homework 5 \\ ECES 511}
\date{}

\begin{document}

\maketitle

\section*{Problem 1}
For the matrix $\mathbf{A}$ given as
$$
\mathbf{A}
=
\begin{bmatrix}
0 & 1 \\
-2 & -3 \\
\end{bmatrix}
$$
Compute the state transition matrix $\varphi (t) = e^{\mathbf{A}t}$ by
\begin{enumerate}[1.]
\item Using the Laplace transform
\newline
$$ e^{\mathbf{A}t} = \mathcal{L}^{-1}(sI-A)^{-1} $$
$$ (sI-A) =
\begin{bmatrix}
s  & -1 \\
2 & s+3
\end{bmatrix}
\implies
(sI-A)^{-1}
=
\frac{1}{(s+1)(s+2)}
\begin{bmatrix}
s+3  & 1 \\
-2 & s
\end{bmatrix}
$$
$$
\mathcal{L}^{-1}(sI-A)^{-1}
=
\begin{bmatrix}
2e^{-t}-e^{-2t}  & e^{-t}-e^{-2t} \\
2e^{-2t}-2e^{-t} & 2e^{-2t}-e^{-t}
\end{bmatrix}
$$

\item Using Cayley Hamilton's theorem
\newline
Let $f(\lambda) = e^{\lambda t}$
$$\Delta(A) = \lambda^2 + 3\lambda + 2 = (\lambda+2)(\lambda+1)$$
$$ \lambda_1 = -1 \qquad \lambda_2 = -2 $$
Let $h(\lambda) = \beta_0 + \beta_1\lambda$
\newline
With $\lambda_1$ and knowing it's multiplicity is $1$ we have one equation
$$ h(\lambda_1) = f(\lambda_1) \implies \beta_0 + \beta_1(-1) = e^{-t}$$
With $\lambda_2$ and knowing it's multiplicity is $1$ we have another equation
$$ h(\lambda_2) = f(\lambda_2) \implies \beta_0 + \beta_1(-2) = e^{-2t}$$
$$ \beta_0 = e^{-2t}(2e^{t}-1)  \qquad \beta_1 = e^{-2t}(e^t-1) $$
$$
e^{\mathbf{A}t} = f(\mathbf{A}) = h(\mathbf{A}) =
\beta_1\mathbf{A} + \beta_0
=
(e^{-2t}(e^t-1))\mathbf{A} + e^{-2t}(2e^{t}-1)\mathbf{I}
$$
$$
e^{\mathbf{A}t}
=
\begin{bmatrix}
2e^{-t}-e^{-2t}  & e^{-t}-e^{-2t} \\
2e^{-2t}-2e^{-t} & 2e^{-2t}-e^{-t}
\end{bmatrix}
$$
\item Determining the Jordan form $\mathbf{J}$ of $\mathbf{A}$ and computing $e^{\mathbf{A}t} = \mathbf{Q}e^{\mathbf{J}t}\mathbf{Q}^{-1}$
\newline
We already found $\lambda_1 = -1 \quad \lambda_2 = -2$ so we just need to find the associated eigenvectors.
$$
(-1I -A)q_1=0
=
\begin{bmatrix}
-1 & -1 \\
2 & 2
\end{bmatrix}
q_1=0
\implies
q_1 = \begin{bmatrix} 1 \\ -1 \end{bmatrix}
$$
$$
(-2I -A)q_2=0
=
\begin{bmatrix}
-2 & -1 \\
2 & 1
\end{bmatrix}
q_2=0
\implies
q_2 = \begin{bmatrix} 1 \\ -2 \end{bmatrix}
$$
We can now generate our diagonalized matricies
$$
\mathbf{Q}=\begin{bmatrix} q_1 & q_2 \end{bmatrix}
\qquad
\mathbf{J}
=
\begin{bmatrix}
\lambda_1 & 0 \\
0 & \lambda_2
\end{bmatrix}
\qquad
Q^{-1}
=\frac{1}{det(\mathbf{Q})} adj(\mathbf{Q})
$$
$$
\mathbf{Q}=\begin{bmatrix} 1 & 1 \\ -1  & -2 \end{bmatrix}
\qquad
\mathbf{J}
=
\begin{bmatrix}
-1 & 0 \\
0 & -2
\end{bmatrix}
\qquad
Q^{-1}
= -
\begin{bmatrix}
2 & 1\\
-1 & -1
\end{bmatrix}
$$
$$
e^{\mathbf{A}t}
=\mathbf{Q} e^{\mathbf{J}t} \mathbf{Q}^{-1}
$$
$$
\mathbf{Q}=
\begin{bmatrix}
1 & 1 \\
-1  & -2
\end{bmatrix}
\begin{bmatrix}
e^{-t} & 0 \\
0 & e^{-2t}
\end{bmatrix}
\begin{bmatrix}
2 & 1\\
-1 & -1
\end{bmatrix}
=
\begin{bmatrix}
1 & 1 \\
-1  & -2
\end{bmatrix}
\begin{bmatrix}
2e^{-t} & e^{-t} \\
-e^{-2t} & -e^{-2t}
\end{bmatrix}
$$
$$
e^{\mathbf{A}t}
=
\begin{bmatrix}
2e^{-t}-e^{-2t}  & e^{-t}-e^{-2t} \\
2e^{-2t}-2e^{-t} & 2e^{-2t}-e^{-t}
\end{bmatrix}
$$

\end{enumerate}

\section*{Problem 2}
For the matrix $\mathbf{J}$ given as

$$
\mathbf{J}
=
\begin{bmatrix}
2 & 1 & 0 & 0 & 0 & 0 & 0 & 0 \\
0 & 2 & 0 & 0 & 0 & 0 & 0 & 0 \\
0 & 0 & -3& 1 & 0 & 0 & 0 & 0 \\
0 & 0 & 0 & -3& 1 & 0 & 0 & 0 \\
0 & 0 & 0 & 0 & -3& 0 & 0 & 0 \\
0 & 0 & 0 & 0 & 0 & -3& 0 & 0 \\
0 & 0 & 0 & 0 & 0 & 0 & 4 & 0 \\
0 & 0 & 0 & 0 & 0 & 0 & 0 & 4 \\
\end{bmatrix}
$$
\begin{enumerate}[1.]
\item Using the relationships in the notes, find $e^{\mathbf{J}t}$;
$$
e^{\mathbf{J}t}
=
\begin{bmatrix}
e^{2t} & te^{2t} & 0 & 0 & 0 & 0 & 0 & 0 \\
0 & e^{2t} & 0 & 0 & 0 & 0 & 0 & 0 \\
0 & 0 & e^{-3t}& te^{-3t} & \frac{1}{2}t^2e^{-3t} & 0 & 0 & 0 \\
0 & 0 & 0 & e^{-3t}& te^{-3t} & 0 & 0 & 0 \\
0 & 0 & 0 & 0 & e^{-3t}& 0 & 0 & 0 \\
0 & 0 & 0 & 0 & 0 & e^{-3t}& 0 & 0 \\
0 & 0 & 0 & 0 & 0 & 0 & e^{4t} & 0 \\
0 & 0 & 0 & 0 & 0 & 0 & 0 & e^{4t} \\
\end{bmatrix}
$$
\newline
\item Identify each Jordan block and its order.
\newline
There are $5$ Jordan blocks:
\begin{enumerate}[1.]
\item $J_1$ has order 2
$$
\begin{bmatrix}
e^{2t} & te^{2t} \\
0 & e^{2t} \\
\end{bmatrix}
$$

\item $J_2$ has order $3$
$$
\begin{bmatrix}
e^{-3t}& te^{-3t} & \frac{1}{2}t^2e^{-3t} \\
0 & e^{-3t}& te^{-3t} \\
0 & 0 & e^{-3t}
\end{bmatrix}
$$

\item $J_3$ has order $1$
$$
\begin{bmatrix}
e^{-3t}
\end{bmatrix}
$$

\item $J_4$ has order $1$
$$
\begin{bmatrix}
e^{4t}
\end{bmatrix}
$$

\item $J_5$ has order $1$
$$
\begin{bmatrix}
e^{4t}
\end{bmatrix}
$$
\end{enumerate}

\end{enumerate}

\section*{Problem 3}
Supose we are given a set of points $\{(1,1), (2,2), (3,1.7), (3.5, 2.5), (4,3.6), (5,3.6)\}$.
Between a linear model and a parabolic model which one fits these points better?
Why?
\begin{itemize}
\item Linear Model
\newline
$$y = ax+b \implies y = \begin{bmatrix} x & 1 \end{bmatrix} \begin{bmatrix} a \\ b\end{bmatrix} $$
Solve with least squares we find
$$
\begin{bmatrix} a \\ b \end{bmatrix}
=
\begin{bmatrix}
1 &  1 \\
2 &  1 \\
3 &  1 \\
3.5& 1 \\
4 &  1 \\
5 &  1
\end{bmatrix}
^{-1}
\begin{bmatrix}
1 \\
2 \\
1.7\\
2.5\\
3.6\\
3.6
\end{bmatrix}
=
\begin{bmatrix} .671 \\ .331 \end{bmatrix}
$$
To see how good we did we can look at the error by looking at how close our equation is for the given points.
$$
\text{Error} = \sum{(\hat{y} - y)^2} = \Delta{y}^T\Delta{y}
$$
$$
\Delta{y} =
\begin{bmatrix}
1 &  1 \\
2 &  1 \\
3 &  1 \\
3.5& 1 \\
4 &  1 \\
5 &  1
\end{bmatrix}
\begin{bmatrix} .671 \\ .331 \end{bmatrix}
-
\begin{bmatrix}
1 \\
2 \\
1.7\\
2.5\\
3.6\\
3.6
\end{bmatrix}
=
\begin{bmatrix}
.002 \\
-.3269 \\
.6441\\
.1796\\
-.5849\\
.0861
\end{bmatrix}
$$
$$ \text{Error} = \Delta{y}^T\Delta{y}= .9035 $$



\item Parabolic Model
\newline
$$y = ax^2+bx+c \implies y = \begin{bmatrix}x^2 & x & 1 \end{bmatrix} \begin{bmatrix} a \\ b \\ c \end{bmatrix} $$
Solve with least squares we find
$$
\begin{bmatrix} a \\ b \\ c\end{bmatrix}
=
\begin{bmatrix}
1  & 1 &  1 \\
4  & 2 &  1 \\
9  & 3 &  1 \\
3.5^2& 3.5& 1 \\
16 & 4 &  1 \\
25 & 5 &  1
\end{bmatrix}
^{-1}
\begin{bmatrix}
1 \\
2 \\
1.7\\
2.5\\
3.6\\
3.6
\end{bmatrix}
=
\begin{bmatrix} 0.312 \\ .4862 \\ .5514 \end{bmatrix}
$$
To see how good we did we can look at the error by looking at how close our equation is for the given points.
$$
\text{Error} = \sum{(\hat{y} - y)^2} = \Delta{y}^T\Delta{y}
$$
$$
\Delta{y} =
\begin{bmatrix}
1  & 1 &  1 \\
4  & 2 &  1 \\
9  & 3 &  1 \\
3.5^2& 3.5& 1 \\
16 & 4 &  1 \\
25 & 5 &  1
\end{bmatrix}
\begin{bmatrix} 0.312 \\ .4862 \\ .5514 \end{bmatrix}
-
\begin{bmatrix}
1 \\
2 \\
1.7\\
2.5\\
3.6\\
3.6
\end{bmatrix}
=
\begin{bmatrix}
.0688\\
-.3514 \\
.5906\\
.1351\\
-.6049\\
.1618
\end{bmatrix}
$$
$$ \text{Error} = \Delta{y}^T\Delta{y}= .8875 $$
\end{itemize}
Since our error is lower with the parabolic model it is better described within the region of points by it as well.
\end{document}
