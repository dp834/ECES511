\documentclass{article}

\usepackage[margin=.75in]{geometry}
\usepackage{amsmath}
\usepackage{amssymb}
\usepackage[shortlabels]{enumitem}
\usepackage{pgfplots}

\author{Damien Prieur}
\title{Homework 2 \\ ECES 511}
\date{}

\begin{document}

\maketitle

\section*{Problem 1}
Find the basis of the range spaces and null spaces of the matricies below.

$$
\mathbf{A}_1
=
\begin{bmatrix}
0 & 1 & 0 \\
0 & 0 & 0 \\
0 & 0 & 1 \\
\end{bmatrix}
\qquad
\mathbf{A}_2
=
\begin{bmatrix}
4 & 1 & -1 \\
3 & 2 & 0 \\
1 & 1 & 0 \\
\end{bmatrix}
\qquad
\mathbf{A}_3
=
\begin{bmatrix}
1 & 2 & 4 & 4 \\
0 & -1& -2& 2 \\
0 & 0 & 0 & 1 \\
\end{bmatrix}
$$
Basis($\mathbf{A}_1$) trivially independent:
$$
q_1
=
\begin{bmatrix}
1 \\
0 \\
0 \\
\end{bmatrix}
\qquad
q_2
=
\begin{bmatrix}
0 \\
0 \\
1 \\
\end{bmatrix}
$$
Null space($\mathbf{A}_1$):
$$
\begin{bmatrix}
0 & 1 & 0 \\
0 & 0 & 0 \\
0 & 0 & 1 \\
\end{bmatrix}
\begin{bmatrix}
a \\
b \\
c \\
\end{bmatrix}
=
\begin{bmatrix}
0 & b & 0 \\
0 & 0 & 0 \\
0 & 0 & c \\
\end{bmatrix}
=
\begin{bmatrix}
0 \\
0 \\
0 \\
\end{bmatrix}
\implies
\begin{bmatrix}
a \\
0 \\
0 \\
\end{bmatrix}
$$
With this we get
$
n_1
=
\begin{bmatrix}
1 \\
0 \\
0 \\
\end{bmatrix}
$
as the basis of our null space so our null space($\mathbf{A}_1$) = span\{$n_1$\}.
\newline
Basis($\mathbf{A}_2$): If the determinant is nonzero then all columns are independent of each other since the null space only contains $0$.
With this fact and the Dimension - rank = nullity, we know that the rank is 3 so we have 3 independent basis vectors.
$$
\det{\mathbf{A}_2}
=
\begin{vmatrix}
4 & 1 & -1 \\
3 & 2 & 0 \\
1 & 1 & 0 \\
\end{vmatrix}
= -1
\begin{vmatrix}
3 & 2 \\
1 & 1 \\
\end{vmatrix}
-0
\begin{vmatrix}
4 & 1 \\
1 & 1 \\
\end{vmatrix}
+0
\begin{vmatrix}
4 & 1 \\
3 & 2 \\
\end{vmatrix}
=-1
$$
Therefore a set of basis vectors are:
$$
q_1 =
\begin{bmatrix}
4 \\
3 \\
1 \\
\end{bmatrix}
\qquad
q_2 =
\begin{bmatrix}
1 \\
2 \\
1 \\
\end{bmatrix}
\qquad
q_3 =
\begin{bmatrix}
-1\\
0 \\
0 \\
\end{bmatrix}
$$
Since our rank is $3$ and our dimension is also $3$ (full rank) then our null space must be empty (only contains the $0$ vector).
\newline
Basis($\mathbf{A}_3$): by inspection $2 \cdot q_2 = q_3$ and the rest are independent
$$
q_1 =
\begin{bmatrix}
1 \\
0 \\
0 \\
\end{bmatrix}
\qquad
q_2 =
\begin{bmatrix}
2 \\
-1\\
0 \\
\end{bmatrix}
\qquad
q_3 =
\begin{bmatrix}
4 \\
2 \\
1 \\
\end{bmatrix}
$$
Null space($\mathbf{A}_3$):
$$
\begin{bmatrix}
1 & 2 & 4 & 4 \\
0 & -1& -2& 2 \\
0 & 0 & 0 & 1 \\
\end{bmatrix}
\begin{bmatrix}
a \\
b \\
c \\
d \\
\end{bmatrix}
=
\begin{bmatrix}
a & 2b& 4c& 4d\\
0 & -b&-2c& 2d\\
0 & 0 & 0 & 1d\\
\end{bmatrix}
=
\begin{bmatrix}
0 \\
0 \\
0 \\
0 \\
\end{bmatrix}
\implies
\begin{bmatrix}
0 \\
b \\
-\frac{b}{2} \\
0 \\
\end{bmatrix}
$$
With this we get $n_1 = \begin{bmatrix} 0 \\ 1 \\ -\frac{1}{2} \\ 0\end{bmatrix}$ as the basis of our null space so the null space($\mathbf{A}_3$) = span\{$n_1$\}.


\section*{Problem 2}
Let
$\mathbf{v_1}
=
\begin{bmatrix}
1  \\
2  \\
1  \\
-2 \\
3  \\
\end{bmatrix}
$
,
$\mathbf{v_2}
=
\begin{bmatrix}
2  \\
5  \\
-1 \\
3  \\
-2 \\
\end{bmatrix}
$
,
$\mathbf{v_3}
=
\begin{bmatrix}
1  \\
3  \\
-2 \\
5  \\
-5 \\
\end{bmatrix}
$
,
$\mathbf{v_4}
=
\begin{bmatrix}
3  \\
1  \\
2  \\
-4 \\
1  \\
\end{bmatrix}
$
,
$\mathbf{v_5}
=
\begin{bmatrix}
5  \\
6  \\
1  \\
-1 \\
-1 \\
\end{bmatrix}
$
\newline
Let $\mathbf{W}: \{ \mathbf{w}\in \mathbb{R}^5 \quad | \quad \mathbf{w} = a_1\mathbf{v_1} + a_2\mathbf{v_2} + a_3\mathbf{v_3} + a_4\mathbf{v_4} + a_5\mathbf{v_5} \}$
\newline
\begin{enumerate}[1.]
\item Find basis $\mathbf{Q}$ of $\mathbf{W}$
\newline
By inspection we can see that $q_1 + q_3 = q_2$ and $q_1 + q_3 + q_4 = q_5 $ the rest are linearly independent.
So we keep remove 2 of the linearly dependent, I'm choosing $q_2, q_5$ and keeping $q_1,q_3,q_4$
$$
\mathbf{Q}
=
\begin{bmatrix}
1  & 1  & 3  \\
2  & 3  & 1  \\
1  & -2 & 2  \\
-2 & 5  & -4 \\
3  & -5 & 1  \\
\end{bmatrix}
$$
\item What is the dimension of $\mathbf{W}$
\newline
$dim(\mathbf{W}) = 3$
\item What is the representation of vector $\mathbf{x} = \begin{bmatrix}-5 \\ 6 \\ -5 \\ 11 \\ -1 \end{bmatrix}$ with respect to the basis $\mathbf{Q}$
\newline
We can find the coefficients by gauss jordan reduction
$$
\begin{bmatrix}
1  & 1  & 3  \\
2  & 3  & 1  \\
1  & -2 & 2  \\
-2 & 5  & -4 \\
3  & -5 & 1  \\
\end{bmatrix}
\begin{bmatrix}
a_1 \\
a_2 \\
a_3 \\
\end{bmatrix}
=
\begin{bmatrix}
-5 \\
6  \\
-5 \\
11 \\
-1
\end{bmatrix}
\Leftrightarrow
\begin{bmatrix}
1  & 1  & 3  & -5 \\
2  & 3  & 1  & 6  \\
1  & -2 & 2  & -5 \\
-2 & 5  & -4 & 11 \\
3  & -5 & 1  & -1 \\
\end{bmatrix}
$$
$$
\begin{bmatrix}
1  & 1  & 3  & -5 \\
2  & 3  & 1  & 6  \\
1  & -2 & 2  & -5 \\
-2 & 5  & -4 & 11 \\
3  & -5 & 1  & -1 \\
\end{bmatrix}
\implies
\begin{bmatrix}
1  & 1  & 3  & -5 \\
0  & 1  & -5 & 16 \\
0  & -3 & -1 & 0  \\
0  & 7  & 2  & 1  \\
0  & -8 & -8 & 14 \\
\end{bmatrix}
\implies
\begin{bmatrix}
1  & 1  & 3  & -5 \\
0  & 1  & -5 & 16 \\
0  & 0  &-48 & 142\\
0  & 0  & 37 &-111\\
0  & 0  & -16& 48 \\
\end{bmatrix}
$$
$$
\implies
\begin{bmatrix}
1  & 1  & 3  & -5 \\
0  & 1  & -5 & 16 \\
0  & 0  &-48 & 142\\
0  & 0  & 0  &-\frac{37}{24}\\
0  & 0  & 0  &\frac{2}{3}\\
\end{bmatrix}
\implies
\begin{bmatrix}
1  & 1  & 3  & -5 \\
0  & 1  & -5 & 16 \\
0  & 0  &-48 & 142\\
0  & 0  & 0  &-\frac{37}{24}\\
0  & 0  & 0  & 0\\
\end{bmatrix}
\implies
\begin{bmatrix}
1  & 0  & 0  & 0 \\
0  & 1  & 0  & 0 \\
0  & 0  & 1  & 0 \\
0  & 0  & 0  & 1 \\
0  & 0  & 0  & 0 \\
\end{bmatrix}
$$
Since the last column is dependent on itself there is no soultion for $\mathbf{x}$ with respect to the basis $\mathbf{Q}$.
\item Orthogonalize $\mathbf{Q}$ using Gram-Schmidt orthogonalization
\newline
$\mathbf{u}_1 = \mathbf{v}_1 = \begin{bmatrix} 1 \\ 2 \\ 1 \\ -2 \\ 3 \\ \end{bmatrix}$
\newline
$
\mathbf{u}_2 = \mathbf{v}_3 - \frac{\mathbf{u}_1^T\mathbf{v}_3}{\sqrt{\mathbf{u}_1\mathbf{u}_1}}\mathbf{u}_1
=
\begin{bmatrix} 1 \\ 3 \\ -2 \\ 5 \\ -5 \\ \end{bmatrix}
-\frac{1+6-2-10-15}{\sqrt{1+4+1+4+3}}\begin{bmatrix} 1 \\ 2 \\ 1 \\ -2 \\ 3 \\ \end{bmatrix}
=
\begin{bmatrix} 3 \\ 1 \\ 2 \\ -4 \\ 1 \\ \end{bmatrix}
-\frac{-20}{\sqrt{13}}\begin{bmatrix}  1 \\ 2 \\ 1 \\ -2 \\ 3 \\ \end{bmatrix}
\approx
\begin{bmatrix}  2.0526 \\ 5.1053 \\ -0.9474 \\ 2.8947 \\ -1.8421 \\ \end{bmatrix}
$
\newline
$
\mathbf{u}_2 = \mathbf{v}_4 - \frac{\mathbf{u}_1^T\mathbf{v}_4}{\sqrt{\mathbf{u}_1\mathbf{u}_1}}\mathbf{u}_1 - \frac{u_2^Tv_4}{\sqrt{\mathbf{u}_2\mathbf{u}_2}}\mathbf{u}_2
=
\begin{bmatrix} 3 \\ 1 \\ 2 \\ -4 \\ 1 \\ \end{bmatrix}
-\frac{3+2+2+8+3}{\sqrt{13}}\begin{bmatrix}  1 \\ 2 \\ 1 \\ -2 \\ 3 \\ \end{bmatrix}
-\frac{\mathbf{u}_2^T\mathbf{v}_4}{6.5534} \begin{bmatrix}   2.0526 \\ 5.1053 \\ -0.9474 \\ 2.8947 \\ -1.8421 \\  \end{bmatrix}
\approx
\begin{bmatrix} 2.2462 \\ -0.4130 \\ 0.9632 \\ -1.8321 \\ -2.0159 \end{bmatrix}
$
\newline
So we have an orthonormal basis:
$
\mathbf{U}
=
\begin{bmatrix}
1  & 2.0526  & 2.2462  \\
2  & 5.1053  & -0.4130 \\
1  & -0.9474 & 0.9632  \\
-2 & 2.8947  & -1.8321 \\
3  & -1.8421 & -2.0159
\end{bmatrix}
$
\newline
Normalizing the basis we get:
$
\mathbf{e}
=
\begin{bmatrix}
0.2294  & 0.3132  & 0.6099  \\
0.4588  & 0.7790  & -0.1121 \\
0.2294  & -0.1446 & 0.2615  \\
-0.4588 & 0.4417  & -0.4974 \\
0.6882  & -0.2811 & -0.5474
\end{bmatrix}
$

\item Let
$\mathbf{B}
=
\begin{bmatrix}
1 & 2 & 1 & 3 & 5 \\
2 & 5 & 3 & 1 & 6 \\
1 & -2& -2& 2 & 1 \\
-2& 3 & 5 & -4& -1\\
3 & -2& -5& 1 & -1\\
\end{bmatrix}
$
, find the range($\mathbf{B}$), the column space of $\mathbf{B}$, the rank of $\mathbf{B}$ and the null space of $\mathbf{B}$
\newline
We saw from above that $q_2$ and $q_5$ are linearly dependent on the other columns so the range will be the span of the remaining vecotrs.
\newline
range($\mathbf{B}$)=span(
$
\begin{bmatrix}
1  \\
2  \\
1  \\
-2 \\
3  \\
\end{bmatrix}
\begin{bmatrix}
1  \\
3  \\
-2 \\
5  \\
-5 \\
\end{bmatrix}
\begin{bmatrix}
3  \\
1  \\
2  \\
-4 \\
1  \\
\end{bmatrix}
$
)
\newline
The column space of $\mathbf{B}$ is the set of linearly independent vectors
\{$
\begin{bmatrix}
1  \\
2  \\
1  \\
-2 \\
3  \\
\end{bmatrix}
\begin{bmatrix}
1  \\
3  \\
-2 \\
5  \\
-5 \\
\end{bmatrix}
\begin{bmatrix}
3  \\
1  \\
2  \\
-4 \\
1  \\
\end{bmatrix}
$\}
\newline
\newline
Since we have 3 linearly independent vectors we have rank($\mathbf{B}$)=$3$
\newline
\newline
Finding the null space:
$$
\begin{bmatrix}
1  & 2  & 1  & 3  & 5  \\
2  & 5  & 3  & 1  & 6  \\
1  & -1 & -2 & 2  & 1  \\
-2 & 3  & 5  & -4 & -1 \\
3  & -2 & -5 & 1  & -1
\end{bmatrix}
\begin{bmatrix}
a \\
b \\
c \\
d \\
e \\
\end{bmatrix}
=
\begin{bmatrix}
0\\
0\\
0\\
0\\
0\\
\end{bmatrix}
\implies
\begin{bmatrix}
1  & 2  & 1  & 3  & 5  & 0 \\
2  & 5  & 3  & 1  & 6  & 0 \\
1  & -1 & -2 & 2  & 1  & 0 \\
-2 & 3  & 5  & -4 & -1 & 0 \\
3  & -2 & -5 & 1  & -1 & 0 \\
\end{bmatrix}
$$
Row reduce with gaussian elimination:
$$
\begin{bmatrix}
1  & 0  & -1 & 0  & 0  & 0 \\
0  & 1  & 1  & 0  & 1  & 0 \\
0  & 0  & 0  & 1  & 1  & 0 \\
0  & 0  & 0  & 0  & 0  & 0 \\
0  & 0  & 0  & 0  & 0  & 0 \\
\end{bmatrix}
$$
\newline
Free variables $c$ and $e$ which gives us
$n_1 = \begin{bmatrix} 1 \\ -1 \\ 1 \\ 0 \\ 0 \\ \end{bmatrix}$
and
$n_2 = \begin{bmatrix} 0 \\ -1 \\ 0 \\ -1 \\ 1 \\ \end{bmatrix}$
\newline
So the null space is the span($n_1,n_2$)


\end{enumerate}


\end{document}
