\documentclass{article}

\usepackage[margin=.75in]{geometry}
\usepackage{amsmath}
\usepackage{amssymb}
\usepackage[shortlabels]{enumitem}

\author{Damien Prieur}
\title{Special Assignment 1 \\ ECES 511}
\date{}

\begin{document}

\maketitle

This special assignment is designed for linearization and simulation of a single link planar arm with a DC motor.
The model equations which were given in the class:

\begin{align}
\dot{x}_1 &= x_2 \nonumber \\
\dot{x}_2 &= \frac{1}{J}(Nk_tx_3 - bx_2 - mgl\sin(x_1) ) \nonumber \\
\dot{x}_3 &= \frac{1}{L}(u(t) - Rx_3 - k_bNx_2) \nonumber
\end{align}
where
$$
\begin{bmatrix}
x_1 \\
x_2 \\
x_3 \\
\end{bmatrix}
=
\begin{bmatrix}
\theta \\
\dot{\theta} \\
i \\
\end{bmatrix}
$$
\newline
Input $u(t)$ to the system is the motor supply voltage.
\newline
\newline
The System parameters,
$$
J = 0.02,
k_t = k_b = 0.01,
N = 10,
R = 2,
L = 0.5,
b = 0.2,
m = 1,
g = 10,
l = 0.3
$$

\begin{enumerate}[a)]
\item Derive equilibrium states given input voltage $u = 15$, $u=21.2$, $u=30$,$u = 40$, and $u = 75$.
What do you observe?
Does the equilibrium exist for each input?
Justify your answers.
\newline
\newline
We can find the equilibrium state for any input $u(t) = \text{constant}$ by setting all the derivatives to $0$.
\begin{align}
0 = \dot{x}_1 &= x_2 \nonumber \\
0 = \dot{x}_2 &= \frac{1}{J}(Nk_tx_3 - bx_2 - mgl\sin(x_1) ) \nonumber \\
0 = \dot{x}_3 &= \frac{1}{L}(u(t) - Rx_3 - k_bNx_2) \nonumber
\end{align}
By the first equation we can see that $x_2 = 0$. Substituting we get
\begin{align}
0 = \dot{x}_2 &= \frac{1}{J}(Nk_tx_3 - mgl\sin(x_1) ) \nonumber \\
0 = \dot{x}_3 &= \frac{1}{L}(u(t) - Rx_3) \nonumber
\end{align}
By the third equation we have $x_3 = \frac{u(t)}{R}$ Substituting we get
\begin{align}
0 = \dot{x}_2 = \frac{1}{J}(Nk_t\frac{u(t)}{R}- mgl\sin(x_1) ) \nonumber
\end{align}
By the second equation we get
$$
\sin(x_1) = \frac{Nk_tu(t)}{Rmgl}
$$

Combining these equations we get that equilibria for $u(t) = c$ for $c\in \mathbb{R}$
$$
\begin{bmatrix}
x_1 \\
x_2 \\
x_3 \\
\end{bmatrix}
=
\begin{bmatrix}
\sin^{-1}(\frac{Nk_tu(t)}{Rmgl}) \\
0 \\
\frac{u(t)}{R}\\
\end{bmatrix}
$$
Plugging in for each value of $u$ we get
$$
u(t)=15 \implies x \approx \begin{bmatrix} 0.25268 \\ 0 \\ 7.5 \end{bmatrix}
\qquad
u(t)=21.2 \implies x \approx \begin{bmatrix} 0.361132 \\ 0 \\ 10.6 \end{bmatrix}
$$
$$
u(t)=30 \implies x \approx \begin{bmatrix} 0.523599 \\ 0 \\ 15 \end{bmatrix}
\qquad
u(t)=40 \implies x \approx \begin{bmatrix} 0.729728 \\ 0 \\ 20 \end{bmatrix}
$$
$$
u(t)=75 \implies x \approx \begin{bmatrix} 1.5708 - 0.693147i \\ 0 \\ 32.5 \end{bmatrix}
$$

In general as $u(t)$ increases so does the equilibrium angle (shown in radians) and it never passes $\frac{\pi}{2}$.
The current also linearly increases which we expect due to $V=IR$.
An equilibrium point can be calculated for each, but if the equilibrium point doesn't fall in $[-\frac{\pi}{2}, \frac{\pi}{2}]$ then we get complex values.
This is because no real feasible equilibrium exists, the arm will continue to spin.
For the $u(t)=75$ example the motor is providing more torque than the weight at the point of highest torque so it will never stop rotating.
\newline
Side-note: each solutions has a second equilibrium point (infinitely many since we have a circle) that is outside the $[-\frac{\pi}{2}, \frac{\pi}{2}]$ range.
These points are semi-stable equilibrium as from one direction they attract (coming from points in the stable range) and repel points on the other side.
This is consistent with what we expect if we look at the torque  when the arm is above the horizontal axis.

\item Derive linear models around equilibrium points for $u = 15$ volts and $u = 45$ volts.
This will give you two separate linearized systems which you will need to convert into state equations.
Hint: Look up the ss command in MATLAB.
\newline
Performing linearization around an equilibrium point $x_0 = \begin{bmatrix} x_{01} \\ x_{02} \\ x_{03} \end{bmatrix} $ with input $u(t) = u_0$ we get

$$
A = \frac{\partial h}{\partial x}
=
\begin{bmatrix}
0 & 1 & 0 \\
\frac{-mgl\cos{x_1}}{J} & \frac{-b}{J} & \frac{Nk_t}{J} \\
0 & \frac{-k_bN}{L} & \frac{-R}{L} \\
\end{bmatrix}
\qquad
B = \frac{\partial h}{\partial u}
\begin{bmatrix}
0 \\
0 \\
\frac{1}{L} \\
\end{bmatrix}
\qquad
C =
\begin{bmatrix}
1 \\
0 \\
0 \\
\end{bmatrix}
\qquad
D =
\begin{bmatrix}
0 \\
\end{bmatrix}
$$

$$
u(t) = 15
\qquad
A
=
\begin{bmatrix}
0 & 1 & 0 \\
-145.2369 & -10 & 5 \\
0 & -0.2 & -4 \\
\end{bmatrix}
\qquad
\begin{bmatrix}
0 \\
0 \\
2 \\
\end{bmatrix}
\qquad
C =
\begin{bmatrix}
1 \\
0 \\
0 \\
\end{bmatrix}
\qquad
D =
\begin{bmatrix}
0 \\
\end{bmatrix}
$$

$$
u(t) = 45
\qquad
A
=
\begin{bmatrix}
0 & 1 & 0 \\
-99.2157 & -10 & 5 \\
0 & -0.2 & -4 \\
\end{bmatrix}
\qquad
\begin{bmatrix}
0 \\
0 \\
2 \\
\end{bmatrix}
\qquad
C =
\begin{bmatrix}
1 \\
0 \\
0 \\
\end{bmatrix}
\qquad
D =
\begin{bmatrix}
0 \\
\end{bmatrix}
$$

\item Simulate the linearized systems using MATLABS $lsim$ command.
Plot (you will need your state space models from part (b) for this) the state trajectories over a variety of different initial conditions for each of your linearized models.
\begin{enumerate}[i.]
\item $u = 10, x_0 = \begin{bmatrix} \frac{\pi}{2} & 0 & 0 \end{bmatrix}$
\item $u = 21.2, x_0 = \begin{bmatrix} 0 & 0 & 0 \end{bmatrix}$
\item $u = 21.2, x_0 = \begin{bmatrix} \frac{\pi}{4} & 0 & 0 \end{bmatrix}$
\item $u = 21.2, x_0 = \begin{bmatrix} \frac{\pi}{3} & 0 & 0 \end{bmatrix}$
\item $u = 21.2, x_0 = \begin{bmatrix} \frac{\pi}{2} & 0 & 0 \end{bmatrix}$
\item $u = 30, x_0 = \begin{bmatrix} -\frac{\pi}{2} & 0 & 0 \end{bmatrix}$
\item $u = 30, x_0 = \begin{bmatrix} 0 & 0 & 0 \end{bmatrix}$
\end{enumerate}

\item Repeat the exact same input and initial conditions used in part (c) but this time using the non-linear model and MATLAB's $ode45$ command.
Compare your results from parts C and D by plotting the trajectories on the same figure.
Play around with this and try different inputs with your three systems ( non-linear and two linear models).
How are each of the models behaving?
For example, what happens if you use the 45 volt linearized model and give an input of 5 volts Vs 50 volts?
What about 80 volts?
Compare this to the non-linear model on the same plot?
\newline

\item Compare (by plotting the trajectories on the same figure so that it is easy to compare) and discuss the results explaining the benefits and risks of the local linearization method.
\newline

\item Repeat the exact same input and initial conditions simulation for 4-5 seconds duration using the Rectangular, Trapazoidal, and Runge Kutta numerical approximation techniques.
Compare and plot all three techniques on the same graph Vs the nonlinear model using $ode45$.
\newline

\item Compare (by plotting the trajectories on the same figure so that it is easy to compare) and discuss the results explaining the difference between numerical methods such as accuracy and efficiency.
\newline

\end{enumerate}


\end{document}
