\documentclass{article}

\usepackage[margin=.75in]{geometry}
\usepackage{amsmath}
\usepackage{amssymb}
\usepackage[shortlabels]{enumitem}
\usepackage{pgfplots}

\author{Damien Prieur}
\title{Homework 3 \\ ECES 511}
\date{}

\begin{document}

\maketitle

\section*{Problem 1}
\begin{enumerate}[a)]
\item Consider the linear system whose augmented matrix is given by
$
\left[
\begin{array}{ccc|c}
1 &  1 & 2 & 1 \\
1 & -2 & 3 & b \\
2 &  0 & b & 0 \\
\end{array}
\right]
$
, where $b$ is a real number.
For what number $b$ will the system have a unique solution?
\newline
\newline
We know that for a unique solution to exist the determinant must exist so what values of b are not allowed?
Expand along the third row and we get
$$
\begin{vmatrix}
1 &  1 & 2 \\
1 & -2 & 3 \\
2 &  0 & b \\
\end{vmatrix}
=
 2 * \begin{vmatrix} 1 & 2 \\ -2 & 3\\ \end{vmatrix}
-0 * \begin{vmatrix} 1 & 2 \\  1 & 3\\ \end{vmatrix}
+b * \begin{vmatrix} 1 & 1 \\  1 &-2\\ \end{vmatrix}
= 2*(3+4) + b(-2 -1)
=14 - 3b
$$
This implies that $b \neq \frac{14}{3}$ and all other values of $b$ should have a unique solution.





\item What can you say about the number of solutions to the system for other values of $b$?
\newline
\newline
This means that for $b=\frac{13}{4}$ there is either no solution or infinitely many solutions, in our case there is at least one solution therefore there are infinitely many solutions.

\end{enumerate}

\section*{Problem 2}
Manually find the matrix $\mathbf{Q}$ that will diagonalize the matrix
$
\mathbf{A}
=
\begin{bmatrix}
0 & 1 & 0 \\
0 & 0 & 1 \\
-6&-11&-6 \\
\end{bmatrix}
$
and compute the diagonalized form.
Validate your solution with MATLAB using {\bf [V,D] = eig(A)}
\newline
\newline
We can start by finding the eigenvalues and eigenvectors of $\mathbf{A}$;
$$
\begin{vmatrix}
\lambda & 1 & 0 \\
0 & \lambda & 1 \\
-6&-11&\lambda-6 \\
\end{vmatrix}
=\lambda (\lambda(\lambda-6) + 11) - 1(6)
=\lambda^3-6\lambda^2+11\lambda -6
=(\lambda-1)(\lambda-2)(\lambda-3)
$$
$$
\lambda_1 = 1 \quad
\lambda_2 = 2 \quad
\lambda_3 = 3
$$
Since we have 3 eigenvalues for our 3 dimensional matrix our diagonal matrix is
$$
\mathbf{\hat{A}}
=
\begin{bmatrix}
1 & 0 & 0 \\
0 & 2 & 0 \\
0 & 0 & 3 \\
\end{bmatrix}
$$
To find $\mathbf{Q}$ we need to find the 3 eigenvectors
$$
\mathbf{q}_1
=
\left[
\begin{array}{ccc|c}
1 & 1 & 0  & 0\\
0 & 1 & 1  & 0\\
-6&-11& -5 & 0\\
\end{array}
\right]
=
\begin{bmatrix}
1 \\
-1 \\
1 \\
\end{bmatrix}
$$
$$
\mathbf{q}_2
=
\left[
\begin{array}{ccc|c}
2 & 1 & 0  & 0\\
0 & 2 & 1  & 0\\
-6&-11& -4 & 0\\
\end{array}
\right]
=
\begin{bmatrix}
1 \\
-2 \\
4 \\
\end{bmatrix}
$$
$$
\mathbf{q}_3
=
\left[
\begin{array}{ccc|c}
3 & 1 & 0  & 0\\
0 & 3 & 1  & 0\\
-6&-11& -3 & 0\\
\end{array}
\right]
=
\begin{bmatrix}
1 \\
-3 \\
6 \\
\end{bmatrix}
$$
Which are trivially solved due to the first two rows.
With our three eigenvectors we can find our $\mathbf{Q}$ matrix
$$
\mathbf{Q}
=
\begin{bmatrix}
1 & 1 & 1 \\
-1&-2 &-3 \\
1 & 4 & 9 \\
\end{bmatrix}
$$

\section*{Problem 3}
Consider the companion-form matrix
$$
\mathbf{A}
=
\begin{bmatrix}
-\alpha_1 & -\alpha_2 & -\alpha_3 & -\alpha_4 \\
1 & 0 & 0 & 0 \\
0 & 1 & 0 & 0 \\
0 & 0 & 1 & 0 \\
\end{bmatrix}
$$
Show that its characteristic polynomial is given by
$$\Delta (\lambda) = \lambda^4 + \alpha_1 \lambda^3 + \alpha_2 \lambda^2 + \alpha_3 \lambda + \alpha_4 $$
Show that if $\lambda_1$ is an eigenvalue of $\mathbf{A}$ or a solution of $\Delta (\lambda) = 0 $ then
$[ \lambda_i^3, \lambda_i^2, \lambda_i, 1]^T$ is an eigenvector of $\mathbf{A}$ associated with $\lambda_i$.
\newline
\newline
$$
\begin{vmatrix}
\lambda +\alpha_1 & -\alpha_2 & -\alpha_3 & -\alpha_4 \\
1 & \lambda & 0 & 0 \\
0 & 1 & \lambda & 0 \\
0 & 0 & 1 & \lambda \\
\end{vmatrix}
$$
Expanding along the last row we get
$$
-1
\begin{vmatrix}
\lambda+\alpha_1 & -\alpha_2 & -\alpha_4 \\
1 & \lambda & 0 \\
0 & 1 & 0 \\
\end{vmatrix}
+ \lambda
\begin{vmatrix}
\lambda + \alpha_1 & -\alpha_2 & -\alpha_3 \\
1 & \lambda & 0 \\
0 & 1 & \lambda \\
\end{vmatrix}
=
-1(-1\alpha_4) + \lambda(-1(+\alpha_3) -+\lambda((\lambda+\alpha_1)\lambda + \alpha_2)
= \lambda^4 + \alpha_1 \lambda^3 + \alpha_2 \lambda^2 + \alpha_3 \lambda + \alpha_4
$$
Using $\lambda_1$ an eigenvalue of $\mathbf{A}$ we get an eigenvector
$$
\left[
\begin{array}{cccc|c}
\alpha_1 - \lambda_1 & -\alpha_2 & -\alpha_3 & -\alpha_4  & 0\\
1 & -\lambda_1 & 0 & 0 & 0 \\
0 & 1 & -\lambda_1 & 0 & 0 \\
0 & 0 & 1 & -\lambda_1 & 0 \\
\end{array}
\right]
$$
Solving using back substitution starting from the bottom row we get
$$
\mathbf{q}_1
=
\begin{bmatrix}
\lambda_i^3 \\
\lambda_i^2 \\
\lambda_i^1 \\
1 \\
\end{bmatrix}
$$


\section*{Problem 4}
Show that the companion-form matrix in Problem 3 is nonsingular if and only if $\alpha_4 \neq 0$
Under this assumption show that its inverse equals
$$
\mathbf{A}^{-1}
=
\begin{bmatrix}
0 & 1 & 0 & 0 \\
0 & 0 & 1 & 0 \\
0 & 0 & 0 & 1 \\
-\dfrac{1}{\alpha_4} & -\dfrac{\alpha_1}{\alpha_4} & -\dfrac{\alpha_2}{\alpha_4} & -\dfrac{\alpha_3}{\alpha_4} \\
\end{bmatrix}
$$
\newline
\newline
The equation from problem 3 is trivially singular if $\alpha_4$ is zero since the last column would be the zero vector.
This means it is linearly dependent and therefore the matrix is not full rank.
\newline
So show that the given matrix is the inverse if $\mathbf{A}$ we can look at the product $\mathbf{A} \mathbf{A}^{-1}$.
If it is the inverse we should get the identity matrix.

$$
\begin{bmatrix}
-\alpha_1 & -\alpha_2 & -\alpha_3 & -\alpha_4 \\
1 & 0 & 0 & 0 \\
0 & 1 & 0 & 0 \\
0 & 0 & 1 & 0 \\
\end{bmatrix}
\begin{bmatrix}
0 & 1 & 0 & 0 \\
0 & 0 & 1 & 0 \\
0 & 0 & 0 & 1 \\
-\dfrac{1}{\alpha_4} & -\dfrac{\alpha_1}{\alpha_4} & -\dfrac{\alpha_2}{\alpha_4} & -\dfrac{\alpha_3}{\alpha_4} \\
\end{bmatrix}
=
\begin{bmatrix}
1 & 0 & 0 & 0 \\
0 & 1 & 0 & 0 \\
0 & 0 & 1 & 0 \\
0 & 0 & 0 & 1 \\
\end{bmatrix}
$$
Everything does work out and cancel to give the identity matrix, therefore $\mathbf{A}^{-1}$ is the inverse of $\mathbf{A}$.

\section*{Problem 5}
Manually compute the eigenvectors and generalized eigenvectors for the following matrix
\begin{enumerate}[a)]
\item Using the bottom up approach
\newline
\newline
First we compute the eigenvectors and the generalized eigenvectors.
$$
\begin{bmatrix}
3-\lambda & 2 & 3 \\
0 & 3-\lambda & 4 \\
0 & 0 & 3-\lambda \\
\end{bmatrix}
=
(3-\lambda)(3-\lambda)(3-\lambda)
$$
Our eigenvalues are a triple root $\lambda_{1,2,3} = 3$. Solving for our eigenvectors
$$
\left[
\begin{array}{ccc|c}
3-3 & 2 & 3 & 0 \\
0 & 3-3 & 4 & 0 \\
0 & 0 & 3-3 & 0 \\
\end{array}
\right]
\implies
\mathbf{q}_1
=
\begin{bmatrix}
1 \\
0 \\
0 \\
\end{bmatrix}
$$
$$
\left[
\begin{array}{ccc|c}
0 & 2 & 3 & 1 \\
0 & 0 & 4 & 0 \\
0 & 0 & 0 & 0 \\
\end{array}
\right]
=
\mathbf{q}_2
=
\begin{bmatrix}
0 \\
\frac{1}{2} \\
0 \\
\end{bmatrix}
$$
$$
\left[
\begin{array}{ccc|c}
0 & 2 & 3 & 0 \\
0 & 0 & 4 & \frac{1}{2} \\
0 & 0 & 0 & 0 \\
\end{array}
\right]
=
\mathbf{q}_3
=
\begin{bmatrix}
0 \\
\frac{-3}{16} \\
\frac{1}{8}\\
\end{bmatrix}
$$
Combining these we get
$$
\mathbf{Q}
=
\begin{bmatrix}
1 & 0 & 0 \\
0 & \frac{1}{2} & \frac{-3}{16} \\
0 & 0 & \frac{1}{8} \\
\end{bmatrix}
$$

\item Using the top down approach
\newline
\newline
From the same problem as before we know our first eigenvector is
$$
\mathbf{q}_1
=
\begin{bmatrix}
1 \\
0 \\
0 \\
\end{bmatrix}
$$
Next solve
$$(\mathbf{A} - \lambda\mathbf{I})^2
\implies
\begin{bmatrix}
3-3 & 2 & 3 \\
0 & 3-3 & 4 \\
0 & 0 & 3-3 \\
\end{bmatrix}
\begin{bmatrix}
3-3 & 2 & 3 \\
0 & 3-3 & 4 \\
0 & 0 & 3-3 \\
\end{bmatrix}
=
\begin{bmatrix}
0 & 0 & 8 \\
0 & 0 & 0 \\
0 & 0 & 0 \\
\end{bmatrix}
$$

$$(\mathbf{A} - \lambda\mathbf{I})^3
\implies
=
\begin{bmatrix}
0 & 0 & 0 \\
0 & 0 & 0 \\
0 & 0 & 0 \\
\end{bmatrix}
$$
Choose $\mathbf{q} = \begin{bmatrix} 1 \\ 1 \\ 1 \end{bmatrix}$. We get
$$
\mathbf{q}_3 = \mathbf{q}
$$
$$
\mathbf{q}_2 = (\mathbf{A} -\lambda\mathbf{I})\mathbf{q} = 0 \implies
\begin{bmatrix}
0 & 2 & 3 \\
0 & 0 & 4 \\
0 & 0 & 0 \\
\end{bmatrix}
\begin{bmatrix}
1\\
1\\
1\\
\end{bmatrix}
=
\begin{bmatrix}
5\\
4\\
0\\
\end{bmatrix}
$$
$$
\mathbf{q}_1 = (\mathbf{A} -\lambda\mathbf{I})^2\mathbf{q} = 0 \implies
\begin{bmatrix}
0 & 0 & 8 \\
0 & 0 & 0 \\
0 & 0 & 0 \\
\end{bmatrix}
\begin{bmatrix}
1\\
1\\
1\\
\end{bmatrix}
=
\begin{bmatrix}
8\\
0\\
0\\
\end{bmatrix}
$$
$$
\mathbf{q}_0 = (\mathbf{A} -\lambda\mathbf{I})^3\mathbf{q} = 0
$$
Combining these we get that
$$
\mathbf{Q}
=
\begin{bmatrix}
8 & 5 & 1 \\
0 & 4 & 1 \\
0 & 0 & 1 \\
\end{bmatrix}
$$

\item Compute the Jordan form from the similarity transform $J = Q^{-1}AQ$
$$
\mathbf{A}
=
\begin{bmatrix}
3 & 2 & 3 \\
0 & 3 & 4 \\
0 & 0 & 3 \\
\end{bmatrix}
$$
\newline
\newline
$$
\mathbf{J}
=
\mathbf{Q}^{-1}
\mathbf{A}
\mathbf{Q}
=
\begin{bmatrix}
1 & 0 & 0 \\
0 & 2 & 3 \\
0 & 0 & 8 \\
\end{bmatrix}
\begin{bmatrix}
3 & 2 & 3 \\
0 & 3 & 4 \\
0 & 0 & 3 \\
\end{bmatrix}
\begin{bmatrix}
1 & 0 & 0 \\
0 & \frac{1}{2} & \frac{-3}{16} \\
0 & 0 & \frac{1}{8} \\
\end{bmatrix}
=
\begin{bmatrix}
3 & 1 & 0 \\
0 & 3 & 1 \\
0 & 0 & 3 \\
\end{bmatrix}
$$
\end{enumerate}
Validate your work with Matlab using [u,v]=jordan(A)

\end{document}
