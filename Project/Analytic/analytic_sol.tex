\documentclass{beamer}
\usetheme{Rochester}
\usecolortheme{beetle}

\usepackage[shortlabels]{enumitem}
\usepackage{graphicx}
\usepackage{amsmath}
\usepackage{mathtools}
\usepackage[b]{esvect}
\usepackage{tikz}
\usepackage{float}
\usepackage{subcaption}

%Information to be included in the title page:
\title{Analytic Solution}
\subtitle{ECES 511}
\author{Damien Prieur}
\date{11/10/2020}

\begin{document}

\frame{\titlepage}

\begin{frame}
\frametitle{Electromechanical Magnetic-Ball Suspension}
\begin{itemize}[$\bullet$]
\item Make an object levitate by controlling the current
\end{itemize}
\begin{figure}
\includegraphics[scale=.75]{"../images/Electromechanical Magnetic-Ball Suspension setup".png}
\end{figure}
\end{frame}

\begin{frame}
\frametitle{The System}
\begin{itemize}
\item $R = 1 \Omega $
\item $L = 0.01 \text{H}$
\item $m = 0.05 \text{kg}$
\item $K = 0.0001 \frac{\text{kg}\cdot\text{m}^2}{\text{A}^2\cdot\text{s}^2}$
\item $g = 9.81 \frac{\text{m}}{\text{s}^2}$
\end{itemize}
$$ v(t) = Ri(t) + L \frac{di(t)}{dt} $$
$$ \frac{d^2y(t)}{dt^2} = mg - K \frac{i^2(t)}{y(t)} $$
$$
\begin{bmatrix}
\dot{x}_1 \\
\dot{x}_2 \\
\dot{x}_3
\end{bmatrix}
=
\begin{bmatrix}
\frac{u(t)-Rx_1(t)}{L} \\
x_3(t) \\
-\frac{Kx_1^2(t)}{mx_2(t)} + g
\end{bmatrix}
$$
$$
y(t) = x_2(t)
$$
\end{frame}

\begin{frame}
\frametitle{Equilibrium}
$$
\begin{bmatrix}
\dot{x}_1 \\
\dot{x}_2 \\
\dot{x}_3
\end{bmatrix}
=
\begin{bmatrix}
0 \\
0 \\
0 \\
\end{bmatrix}
=
\begin{bmatrix}
\frac{u(t)-Rx_1(t)}{L} \\
x_3(t) \\
-\frac{Kx_1^2(t)}{mx_2(t)} + g
\end{bmatrix}
$$
Which gives us
$$ x_1(t) = \frac{u(t)}{R}, \qquad x_3(t) = 0, \qquad x_2(t) = \frac{Ku^2(t)}{mgR^2} $$
\begin{minipage}{.45\textwidth}
    For a constant input voltage we get
    $$ y_0 = \frac{Kv_0^2}{mgR^2} $$
\end{minipage}
\hfill
\begin{minipage}{.45\textwidth}
    If we want to hold the ball at a height $y_0$
    $$ v_0 = \sqrt{\frac{y_0mgR^2}{K}} $$
\end{minipage}%
\end{frame}

\begin{frame}
\frametitle{Linearization about the Equilibrium}
$$
\begin{bmatrix}
\dot{x}_1 \\
\dot{x}_2 \\
\dot{x}_3
\end{bmatrix}
=
\begin{bmatrix}
\frac{u(t)-Rx_1(t)}{L} \\
x_3(t) \\
-\frac{Kx_1^2(t)}{mx_2(t)} + g
\end{bmatrix}
$$
$$
A
=
\frac{\partial h}{\partial x}
=
\begin{bmatrix}
-\frac{R}{L} & 0 & 0\\
0 & 0 & 1\\
-\frac{2Kx_1(t)}{mx_2(t)} & \frac{K}{m}(\frac{x_1(t)}{x_2(t)})^2 & 0\\
\end{bmatrix}
=
\begin{bmatrix}
-\frac{R}{L} & 0 & 0\\
0 & 0 & 1\\
-\frac{2Kx_{01}}{mx_{02}} & \frac{K}{m}(\frac{x_{01}}{x_{02}})^2 & 0\\
\end{bmatrix}
$$
$$
B
=
\frac{\partial h}{\partial u}
=
\begin{bmatrix}
\frac{1}{L} \\
0 \\
0 \\
\end{bmatrix}
\qquad
C
=
\begin{bmatrix}
0 & 1 & 0
\end{bmatrix}
\qquad
D
=
\begin{bmatrix}
0
\end{bmatrix}
$$
\end{frame}

\begin{frame}
\frametitle{Sample Equilibria}
$$
v_0 = 7 \qquad
x_0
=
\begin{bmatrix}
7 \\
.00998 \\
0
\end{bmatrix}
\qquad
A
=
\begin{bmatrix}
-100 & 0 & 0 \\
0 & 0 & 1 \\
-2.803 & 982 & 0
\end{bmatrix}
\qquad
B
=
\begin{bmatrix}
100 \\
0 \\
0
\end{bmatrix}
$$
$$
\mathbf{x}(t)
=
e^{\mathbf{At}}\mathbf{x}(0) + \int_0^t e^{\mathbf{A}(t-\tau)}\mathbf{B}\mathbf{u}(\tau) d\tau
$$
Eigenvalues
$$
(\lambda I - A)x = 0
\quad
\begin{bmatrix}
\lambda + 100 & 0 & 0 \\
0 & \lambda & 1 \\
2.803 & -982 & \lambda
\end{bmatrix}
x = 0
$$
$$
98200 + 982 \lambda - 100 \lambda^2 - \lambda^3
$$
$$
\implies \lambda_1 = -100 \lambda_2 = 31.3369 \quad \lambda_3 = -31.3369
$$
\end{frame}

\begin{frame}
\frametitle{Jordan Decomposition}
$$
(A-\lambda_1 I)q_1=0,
\begin{bmatrix}
0 & 0 & 0 \\
0 & 100 & 1 \\
-2.803 & 982 & 100
\end{bmatrix}
q_1
=
0
\implies
q_1 = \begin{bmatrix} 3217.267 \\ 1 \\ -100 \end{bmatrix}
$$
$$
\begin{bmatrix}
-131.3369 & 0 & 0 \\
0 & -31.3369 & 1 \\
-2.803 & 982 &-31.3369 
\end{bmatrix}
q_2
=0
\implies
q_2 = \begin{bmatrix} 0 \\ 1 \\ 31.3369 \end{bmatrix}
$$
$$
\begin{bmatrix}
-68.6631 & 0 & 0 \\
0 & 31.3369 & 1 \\
-2.803 & 982 & 31.3369
\end{bmatrix}
q_3
=
0
\implies
q_3 = \begin{bmatrix} 0 \\ 1 \\ -31.3369 \end{bmatrix}
$$
\end{frame}

\begin{frame}
\frametitle{Jordan Form}
Normalizing eigenvectors
$$
q_1=
\begin{bmatrix} .999517 \\ -3.1067\times10^{-4} \\ .031067 \end{bmatrix}
\quad
q_2=
\begin{bmatrix} 0 \\ .031895 \\ .999491 \end{bmatrix}
q_3=
\begin{bmatrix} 0 \\ .031895 \\ -.999491 \end{bmatrix}
$$
$$
Q
=
\begin{bmatrix}
.999517             & 0 & 0 \\
-3.1067\times10^{-4} & .031895 & .031895 \\
.031067            & .999491 & -.999491
\end{bmatrix}
$$
$$
J
=
\begin{bmatrix}
-100 & 0 & 0 \\
0 & 31.3369 & 0 \\
0 & 0 & -31.3369
\end{bmatrix}
$$
$$
Q^{-1}
=
\begin{bmatrix}
1.00048 & 0. & 0. \\
-0.0106764 & 15.6764 & 0.500255 \\
 0.0204215 & 15.6764 & -0.500255 \\
\end{bmatrix}
$$
\end{frame}

\begin{frame}
\frametitle{Matrix Exponentiation}
$$ e^{At} = Qe^JQ^{-1}
=
Q
\begin{bmatrix}
e^{-100t} & 0 & 0 \\
0 & e^{31.3369t} & 0 \\
0 & 0 & e^{-31.3369t}
\end{bmatrix}
Q^{-1}
$$
This Expression becomes very large so each column of the matrix will be broken up with the first column followed by the next two.
$$
\begin{bmatrix}
e^{\lambda_1 t} 																								\\
-3.10823\times 10^{-4} e^{\lambda_1 t}+6.51349\times 10^{-4} e^{\lambda_3 t}-3.40526\times 10^{-4} e^{\lambda_2 t} \\
 0.0310823 e^{\lambda_1 t}-0.0204112 e^{\lambda_3 t}-0.010671 e^{\lambda_2 t} 										\\
\end{bmatrix}
$$
$$
\begin{bmatrix}
0 								   			& 0 \\
0.5 e^{\lambda_3 t}+0.5 e^{\lambda_2 t} 			& 0.0159556 e^{\lambda_2 t}-0.0159556 e^{\lambda_3 t} \\
15.6684 e^{\lambda_2 t}-15.6684 e^{\lambda_3 t} 	& 0.5 e^{\lambda_3 t}+0.5 e^{\lambda_2 t} \\
\end{bmatrix}
$$
\end{frame}

\begin{frame}
\frametitle{Step Response}
$$\mathbf{u}(t) = \begin{bmatrix} 1 \end{bmatrix} $$
$$
\mathbf{x}(t)
=
e^{\mathbf{At}}\mathbf{x}(0) + \int_0^t e^{\mathbf{A}(t-\tau)}\mathbf{B}\mathbf{u}(\tau) d\tau
$$
$$
\mathbf{x}(t)
=
e^{\mathbf{A}t}\mathbf{x}(0)
+
$$
$$
\begin{bmatrix}
1 - e^{-100. t} \\
 0.002854\, +3.1082\times 10^{-4} e^{\lambda_1 t}-0.002079 e^{\lambda_3 t}-0.00109 e^{\lambda_2 t} \\
 -0.03108 e^{\lambda_1 t}+0.0651e^{\lambda_3 t}-0.03405 e^{\lambda_2 t} \\
\end{bmatrix}
$$
For a given initial condition
\end{frame}

\begin{frame}
\frametitle{Step Response Analysis}
For a general valid starting condition for the system the following will hold.
\begin{itemize}[$\bullet$]
\item Current term will approach a stable value of $1$ amp
\item Height will drop exponentially as the $\lambda_1$ and $\lambda_2$ terms go to zero while the $\lambda_3$ term corresponds to the constant gravity term which will accelerate the ball downwards until it hits the ground.
\item The vertical velocity term will also continually increase due to gravity.
\end{itemize}
The system is unstable given the point we linearized around with an input $u(t)=1$
\end{frame}

\begin{frame}
\frametitle{Closed Form Solution}
$$\mathbf{x}(0) = \begin{bmatrix} 0 \\ .01 \\ 0 \end{bmatrix} $$
$$\mathbf{x}_1(t) = 1-e^{-100t}$$
$$\mathbf{x}_2(t) = .002854+3.1082\times 10^{-4}e^{-100 t} + .0029215e^{-31.3369 t} +.0039133 e^{31.3369 t} $$
$$\mathbf{x}_3(t) = -.091549e^{-31.3369 t} +.12263 e^{31.3369 t} $$
\end{frame}

\begin{frame}
\frametitle{Step Response Graph}
\begin{figure}[h!]
\centering
	\includegraphics[width=1\linewidth]{{images/linearized_7.000000_input_1.000000_initial_pos_0.010000}.png}
\end{figure}

\end{frame}

\begin{frame}
\frametitle{Step Response Graph}
\begin{itemize}[$\bullet$]
	\item The Current appears to be the correct equation which make sense due to the fact that that part of the system can be described by a linear system.
	\item The Position and velocity graphs appear to be close but not accurate to the simulation. I would attribute this to lack of numerical precision since most of the numbers are quite small and being multiplied making them even smaller.
    Also while computing my close form solution I removed terms that were very close to zero so that I didn't have to carry them through the equations.
\end{itemize}

\end{frame}

\begin{frame}
\frametitle{Poles and Zeros}
$$\mathcal{L}(s) = \mathbf{C}(s\mathbf{I}-\mathbf{A}^{-1})\mathbf{B} $$
$$\mathbf{\hat{x}}(s) = \frac{-280.3}{s^3+100 s^2-982 s-98200} $$
$$ = \frac{-280.3}{(s-31.3369) (s+31.3369) (s+100)}$$
$$\mathcal{L}(u) = \frac{1}{s} $$
$$\mathbf{\hat{y}}(s) = \frac{-280.3}{s(s-31.3369) (s+31.3369) (s+100)}$$
We can see from the factored form that we have no zeros and $4$ poles.
$$ s = \lambda_1 \quad s = \lambda_2 \quad s = \lambda_3 \quad s = 0 $$
\end{frame}
\end{document}
