\documentclass{beamer}
\usetheme{Rochester}
\usecolortheme{beetle}

\usepackage[shortlabels]{enumitem}
\usepackage{graphicx}
\usepackage{amsmath}
\usepackage{mathtools}
\usepackage[b]{esvect}
\usepackage{tikz}
\usepackage{float}
\usepackage{subcaption}

%Information to be included in the title page:
\title{Project Proposal}
\subtitle{ECES 511}
\author{Damien Prieur}
\date{10/1/2020}

\begin{document}

\frame{\titlepage}

\begin{frame}
\frametitle{Electromechanical Magnetic-Ball Suspension}
\begin{itemize}[$\bullet$]
\item Make an object levitate by controlling the current
\end{itemize}
\begin{figure}
\includegraphics[scale=.75]{"../images/Electromechanical Magnetic-Ball Suspension setup".png}
\end{figure}
\end{frame}

\begin{frame}
\frametitle{Mathematical Model}
Variables
\begin{itemize}
\item $R$ - Resistance
\item $L$ - Inductance
\item $v$ - Voltage
\item $m$ - Mass
\item $K$ - Coefficient that relates force to the magnetic field
\item $g$ - Gravity
\item $i$ - Current
\item $y$ - Distance of Mass M to electromagnet
\end{itemize}
$$
\begin{aligned}
    &\begin{aligned}
        \mathllap{v(t)} &= Ri(t) + L\frac{di(t)}{dt}
    \end{aligned} \\
    &\begin{aligned}
        \mathllap{m\frac{d^2y(t)}{dt^2}} &= mg - K\frac{i^2(t)}{y(t)}
    \end{aligned} \\
\end{aligned}
$$
\end{frame}

\begin{frame}
\frametitle{I/O and State Variables}
\begin{itemize}[$\bullet$]
\item We control the current $i$
\item Goal is to control distance $y$
\end{itemize}
$$
\vv{x}=
\begin{bmatrix}
x_1 \\
x_2 \\
x_3
\end{bmatrix}
=
\begin{bmatrix}
i \\
y \\
\dot{y}
\end{bmatrix}
=
\begin{bmatrix}
\text{Current} \\
\text{Distance}\\
\text{Velocity}
\end{bmatrix}
$$

\end{frame}

\begin{frame}
\frametitle{The Actual System}
\begin{itemize}
\item $R = 1 \Omega $
\item $L = 0.01 \text{H}$
\item $m = 0.05 \text{kg}$
\item $K = 0.0001 \frac{\text{kg}\cdot\text{m}^2}{\text{A}^2\cdot\text{s}^2}$
\item $g = 9.81 \frac{\text{m}}{\text{s}^2}$
\end{itemize}
$$ v(t) = Ri(t) + L \frac{di(t)}{dt} $$
$$ \frac{d^2y(t)}{dt^2} = mg - K \frac{i^2(t)}{y(t)} $$
$$
\begin{bmatrix}
\dot{x}_1 \\
\dot{x}_2 \\
\dot{x}_3
\end{bmatrix}
=
\begin{bmatrix}
\frac{u(t)-Rx_1(t)}{L} \\
x_3(t) \\
-\frac{Kx_1^2(t)}{mx_2(t)} + g
\end{bmatrix}
$$
$$
y(t) = x_2(t)
$$
\end{frame}

\begin{frame}
\frametitle{Equilibrium}
$$
\begin{bmatrix}
\dot{x}_1 \\
\dot{x}_2 \\
\dot{x}_3
\end{bmatrix}
=
\begin{bmatrix}
0 \\
0 \\
0 \\
\end{bmatrix}
=
\begin{bmatrix}
\frac{u(t)-Rx_1(t)}{L} \\
x_3(t) \\
-\frac{Kx_1^2(t)}{mx_2(t)} + g
\end{bmatrix}
$$
Which gives us
$$ x_1(t) = \frac{u(t)}{R}, \qquad x_3(t) = 0, \qquad x_2(t) = \frac{Ku^2(t)}{mgR^2} $$
\begin{minipage}{.45\textwidth}
    For a constant input voltage we get
    $$ y_0 = \frac{Kv_0^2}{mgR^2} $$
\end{minipage}
\hfill
\begin{minipage}{.45\textwidth}
    If we want to hold the ball at a height $y_0$
    $$ v_0 = \sqrt{\frac{y_0mgR^2}{K}} $$
\end{minipage}%
\end{frame}

\begin{frame}
\frametitle{Linearization about the Equilibrium}
$$
\begin{bmatrix}
\dot{x}_1 \\
\dot{x}_2 \\
\dot{x}_3
\end{bmatrix}
=
\begin{bmatrix}
\frac{u(t)-Rx_1(t)}{L} \\
x_3(t) \\
-\frac{Kx_1^2(t)}{mx_2(t)} + g
\end{bmatrix}
$$
$$
A
=
\frac{\partial h}{\partial x}
=
\begin{bmatrix}
-\frac{R}{L} & 0 & 0\\
0 & 0 & 1\\
-\frac{2Kx_1(t)}{mx_2(t)} & \frac{K}{m}(\frac{x_1(t)}{x_2(t)})^2 & 0\\
\end{bmatrix}
=
\begin{bmatrix}
-\frac{R}{L} & 0 & 0\\
0 & 0 & 1\\
-\frac{2Kx_{01}}{mx_{02}} & \frac{K}{m}(\frac{x_{01}}{x_{02}})^2 & 0\\
\end{bmatrix}
$$
$$
B
=
\frac{\partial h}{\partial u}
=
\begin{bmatrix}
\frac{1}{L} \\
0 \\
0 \\
\end{bmatrix}
\qquad
C
=
\begin{bmatrix}
0 & 1 & 0
\end{bmatrix}
\qquad
D
=
\begin{bmatrix}
0
\end{bmatrix}
$$
\end{frame}

\begin{frame}
\frametitle{Sample Equilibria}
$$
v_0 = 7 \qquad
x_0
=
\begin{bmatrix}
7 \\
.00998 \\
0
\end{bmatrix}
\qquad
A
=
\begin{bmatrix}
-100 & 0 & 0 \\
0 & 0 & 0 \\
-2.803 & 982 & 0
\end{bmatrix}
$$
\begin{figure}[H]
\begin{subfigure}{.45\textwidth}
    \centering
    \includegraphics[width=1\linewidth]{{images/linearized_7.000000_input_7.000000_initial_pos_0.005000}.png}
\end{subfigure}
\begin{subfigure}{.45\textwidth}
    \centering
    \includegraphics[width=1\linewidth]{{images/linearized_7.000000_input_7.000000_initial_pos_0.009990}.png}
\end{subfigure}
\end{figure}
\end{frame}

\begin{frame}
\frametitle{Sample Equilibria Continued lsim}
\begin{figure}[H]
\begin{subfigure}{.45\textwidth}
    \centering
    \includegraphics[width=1\linewidth]{{images/input_7.000000_initial_pos_0.000000_vs_ode45}.png}
\end{subfigure}
\begin{subfigure}{.45\textwidth}
    \centering
    \includegraphics[width=1\linewidth]{{images/input_10.000000_initial_pos_0.000000_vs_ode45}.png}
\end{subfigure}
\end{figure}
Linearizing the system falls apart if we can't keep the system very close to it's equilibirum point
\newline
\end{frame}

\begin{frame}
\frametitle{Sample Equilibria Continued Numerical Methods}
\begin{figure}[H]
    \centering
    \includegraphics[width=.7\linewidth]{{images/numerical_approx_dt_0.010000_input_7.000000_initial_pos_0.000000_vs_ode45}.png}
\end{figure}
The system becomes very unpredictable when it is near it's equilibrium point, so numerical methods aren't accurate without lowering the step size.
\end{frame}

\begin{frame}
\frametitle{Sample Equilibria Continued Numerical Methods}
\begin{figure}[H]
    \centering
    \includegraphics[width=.7\linewidth]{{images/numerical_approx_dt_0.001000_input_7.000000_initial_pos_0.000000_vs_ode45}.png}
\end{figure}
Even with ten times more datapoints we are still not converging on one result.
The rectangular and trapezoidal methods haven't begun to capture the complexity.
\end{frame}

\begin{frame}
\frametitle{Conclusion}
\begin{itemize}[$\bullet$]
\item This system is extremely unstable and will diverge if it doesn't start in an equilibirum state
\item A linearized model should be accurate if the system can be held close to the equilibrium point
\item Numerical methods do a bad job of approximating the system without decreasing the timestep near its equilibrium point
\item The simulations make sense based on what we can expect from the system since the equilibrium point is not stable
\end{itemize}
\end{frame}

\end{document}
